\documentclass[
    twoside,         % oneside/twoside : Einseitiger oder zweiseitiger Druck?
    12pt,            % Bezug: 12-Punkt Schriftgre
    DIV15,           % Randaufteilung, siehe Dokumentation "KOMA"-Script
    BCOR17mm,        % Bindekorrektur: Innen 17mm Platz lassen. Copyshop-getestet.
    headsepline,     % Unter Kopfzeile Trennlinie (aus: headnosepline)
    footsepline,     % ber Fuzeile Trennlinie (aus: footnosepline)
    openright,       % Neue Kapitel im zweiseitigen Druck rechts beginnen lassen
    a4paper,         % Seitenformat A4
    abstracton,      % Abstract einbinden
    english,
    listof=totoc,version=first,      % Div. Verzeichnisse ins Inhaltsverzeichnis aufnehmen
    bibliography=totoc,version=first,        % Literaturverzeichnis ins Inhaltsverzeichnis aufnehmen
    titlepage,       % Titelseite aktivieren
    headinclude,     % Seiten-Head in die Satzspiegelberechnung mit einbeziehen
    footexclude,     % Seiten-Foot nicht in die Satzspiegelberechnung mit einbeziehen
    numbers=noenddot,version=first % Gliederungsnummern ohne abschlieenden Punkt darstellen
] {scrreprt}       % Dokumentenstil: "Report" aus dem KOMA-Skript-Paket

\usepackage[active]{srcltx}
%\usepackage[activate=normal]{pdfcprot\documentclass[options]{class}} % Optischer Randausgleich -> pdflatex!
\usepackage{ifthen}
\usepackage[english,vietnamese]{babel}
%\usepackage[fixlanguage]{babelbib}
%\setbiblanguage{english}
\usepackage[utf8]{inputenc}
% \usepackage[latin1]{inputenc}
\usepackage[T5,T1]{fontenc}
\usepackage{fontawesome5}
\usepackage[T1]{url}
\usepackage{ae}
\usepackage[final]{graphicx}
\usepackage[automark]{scrlayer-scrpage}
\usepackage{setspace}
\usepackage{subcaption}
\usepackage{floatflt} 
\usepackage{rotating} 
\usepackage{wrapfig}
%\usepackage{subfig}
\usepackage{graphicx}
%\usepackage[first,light]{draftcopy} % Fr Probedruck
\usepackage{lineno}
\usepackage[plainpages=false,pdfpagelabels,hypertexnames=false]{hyperref}
\usepackage{pdfpages} %include pdf files
\usepackage{listings} %include sourcecode
\usepackage{color}
\usepackage{multirow}
\usepackage{verbatim}
\usepackage{amsmath}
\usepackage{amssymb}
\usepackage{amsfonts}
\usepackage{bbm}
\usepackage{enumitem}
\usepackage{array}
\usepackage{nomencl}
\usepackage{xcolor}
\usepackage{xfrac}
\usepackage{longtable}
\usepackage{url}
\usepackage{subcaption}
%\usepackage{paralist}
\usepackage{tipa}
\usepackage{csquotes}
%\usepackage{textcomp}  %mit \textcent geht Cent-Symbol
\usepackage{diagbox}
\usepackage[linesnumbered,ruled]{algorithm2e}
\usepackage{geometry}
\geometry{
    a4paper,
    % total={170mm,257mm},
    left=35mm,
    right=25mm,
    top=20mm,
    bottom=25mm,
}


% Tiefe der Kapitelnummerierung beeinflussen
\setcounter{secnumdepth}{3} % Tiefe der Nummerierung
\setcounter{tocdepth}{3}    % Tiefe des Inhaltsverzeichnisses

% Hier in die zweite geschweifte Klammer jeweils
% die persnlichen Daten eintragen:
\newcommand{\typeofelaboration}{Master Thesis}
\newcommand{\nameofauthor}{Nguyễn Tiến Nam}
\newcommand{\emailofauthor}{hnhat.tran@gmail.com}
\newcommand{\idofauthor}{}
\newcommand{\majorofauthor}{Control Engineering and Automation}

\newcommand*\en{\fontencoding{T1}\selectfont\selectlanguage{english}}
\newcommand*\vn{\fontencoding{T5}\selectfont\selectlanguage{vietnamese}}
\makeatletter
\newcommand{\capitalize}[1]{%
    \edef\@tempa{\expandafter\@gobble\string#1}%
    \edef\@tempb{\expandafter\@car\@tempa\@nil}%
    \edef\@tempa{\expandafter\@cdr\@tempa\@nil}%
    \uppercase\expandafter{\expandafter\def\expandafter\@tempb\expandafter{\@tempb}}%
    \@namedef{\@tempb\@tempa}{\expandafter\MakeUppercase\expandafter{#1}}
}
\makeatother

\newcommand{\markup}[1]{\textbf{#1}}
\newcommand{\github}[1]{\href{#1}{\faGithubSquare\ \textit{#1}}}

% Seitenlayout festlegen. Hier nichts ndern!
\pagestyle{scrplain}
\ihead[]{\scriptsize \headmark}
\ohead[]{\scriptsize \typeofelaboration}
\chead[]{}
% \ifoot[]{\scriptsize \nameofauthor\ - \idofauthor}
\ifoot[]{\scriptsize \nameofauthor}
\ofoot[]{\pagemark}
\cfoot[]{}
\newcommand{\resethead}{\newpage%
    \ihead[]{\scriptsize \headmark}%
    \ohead[]{\scriptsize \typeofelaboration}%
    \chead[]{}%
}
\newcommand{\resetfoot}{\newpage%
    \ifoot[]{\scriptsize \nameofauthor\ - \idofauthor}%
    \ofoot[]{\pagemark}%
    \cfoot[]{}%
}
\renewcommand{\titlepagestyle}{scrheadings}
\renewcommand{\partpagestyle}{scrheadings}
\renewcommand{\chapterpagestyle}{scrheadings}
\renewcommand{\indexpagestyle}{scrheadings}

% Abschnittsweise Nummerierung anstatt fortlaufend. Hier nichts ndern!
\makeatletter
\@addtoreset{equation}{chapter}
\@addtoreset{figure}{chapter}
\@addtoreset{table}{chapter}
\renewcommand\theequation{\thechapter.\@arabic\c@equation}
\renewcommand\thefigure{\thechapter.\@arabic\c@figure}
\renewcommand\thetable{\thechapter.\@arabic\c@table}\makeatother


% Quelltextrahmen, klein. Hier nichts ndern!
\newsavebox{\inhaltkl}
\def\rahmenkl{\sbox{\inhaltkl}\bgroup\small\renewcommand{\baselinestretch}{1}\vbox\bgroup\hsize\textwidth}
\def\endrahmenkl{\par\vskip-\lastskip\egroup\egroup\fboxsep3mm%
\framebox[\textwidth][l]{\usebox{\inhaltkl}}}

% Quelltextrahmen, normale Groesse. Hier nichts ndern!
\newsavebox{\inhalt}
\def\rahmen{\sbox{\inhalt}\bgroup\renewcommand{\baselinestretch}{1}\vbox\bgroup\hsize\textwidth}
\def\endrahmen{\par\vskip-\lastskip\egroup\egroup\fboxsep3mm%
\framebox[\textwidth][l]{\usebox{\inhalt}}}


% Sonstige Befehlsdefinitionen hier ablegen.
\newcommand{\entspricht}{\stackrel{\wedge}{=}}
\definecolor{light-gray}{gray}{0.95}
\addto\captionsenglish{\renewcommand{\contentsname}{Table of Contents}}
%\makenomenclature
\makeglossary
\DeclareMathOperator*{\argmin}{argmin}
\DeclareMathOperator*{\argmax}{argmax}

\newcommand{\listofabbreviations}{%!TEX root = ../main.tex

\cleardoublepage
\ihead[]{\scriptsize{List of Abbreviations}}
\addcontentsline{toc}{chapter}{List of Abbreviations}
\chapter*{List of Abbreviations}

\begin{longtable}{p{.2\textwidth} p{.8\textwidth}}
    \textbf{ANN} & Artificial Neural Network\\
    \textbf{AP} & Average Pooling\\
    \textbf{CCA} & Canonical Correlation Analysis\\
    \textbf{CNN} & Convolutional Neural Network\\
    \textbf{DNN} & Deep Neural Network\\
    \textbf{GRU} & Gated Recurrent Unit\\
    \textbf{HAR} & Human Action Recognition\\
    \textbf{HoG} & Histogram of oriented Gradient\\
    \textbf{iDT} & improved Dense Trajectories\\
    \textbf{KCCA} & Kernel Canonical Correlation Analysis\\
    \textbf{kNN} & k-Neareast Neighbor\\
    \textbf{LDA} & Linear Discriminant Analysis\\
    \textbf{LSTM} & Long Short-Term Memory\\
    \textbf{MICA} & Multimedia, Information, Communication \& Applications International Research Institute\\
    \textbf{MLP} & Multilayer Perceptron\\
    \textbf{MST-AOG} & Multi-view Spatio-Temporal AND-OR Graph\\
    \textbf{MvA} & Multi-view Analysis\\
    \textbf{MvCCA} & Multi-view Canonical Correlation Analysis\\
    \textbf{MvCCDA} & Multi-view Common Component Discriminant Analysis\\
    \textbf{MvDA} & Multi-view Discriminant Analysis\\
    \textbf{MvDA-vc} & Multi-view Discriminant Analysis with View-Consistency\\
    \textbf{MvFDA} & Multi-view Fisher Discriminant Analysis\\
    \textbf{MvMDA} & Multi-view Modular Discriminant Analysis\\
    \textbf{MvML-LA} & Multi-view Manifold Learning with Locality Alignment\\
    \textbf{MvPLS} & Multi-view Partial Least Square\\
    \textbf{pc-LDA} & Pairwise-Covariance Linear Discriminant Analysis\\
    \textbf{pc-MvDA} & Pairwise-Covariance Multi-view Discriminant Analysis\\
    \textbf{RNN} & Recurrent Neural Network\\
    \textbf{SOTA} & State Of The Art\\
    \textbf{SSM} & Self Similarity Matrix\\
    \textbf{TA} & Temporal Attention\\
    % \textbf{SVM} & Support Vector Machine\\
\end{longtable}

\resethead
}
\bibliographystyle{IEEEbib}

\begin{document}

    \onehalfspacing
    \clubpenalty=10000
    \widowpenalty=10000
    \vn
    %!TEX root = ../main.tex

\begin{titlepage}
\thispagestyle{empty}
\begin{center}
\textbf{\large{Hanoi University of Science and Technology}} \\
% \large{School of Electrical Engineering}

\vspace{0.2cm}

\begin{figure}[htbp]
    \begin{center}
    \includegraphics[width=0.2\textwidth]{figs/hust-logo.jpg}
    \end{center}
\end{figure}

\vspace{0.5cm}

\textbf{\Huge{Master Thesis}} \\
\vspace{1.2cm}
\textbf{\LARGE{Human Action Recognition using Deep Learning and Multi-view Discriminant Analysis}} \\

\vspace{1.0cm}

\textbf{\large{TRAN HOANG NHAT}} \\
\small{\emailofauthor} \\
\vspace{0.3cm}
\textbf{\large{\majorofauthor}} \\
\vspace{2.5cm}

\small
\renewcommand{\arraystretch}{1.5} 
\begin{tabular}{lll}
    % \textbf{Submitted by:} & & Tran Hoang Nhat\\
    % \textbf{Date \& place of birth:} & & December 28, 1996, Hanoi\\
    % \textbf{Submission date:} & & October 20, 2020 \\
    % & & \\
    \textbf{Advisor:} & & Assoc. Prof. Dr. Tran Thi Thanh Hai \\
    \textbf{Faculty:} & & School of Electronics and Telecommunications \\
\end{tabular}
%\begin{tabular}{lll}
%\textbf{Faculty Advisors:}  & & Advisor 1 \\ & & Advisor 2 \\ & & Advisor 3\\
% \end{tabular}
% \end{flushleft}

\vspace{\fill}
\textbf{\normalsize{Hanoi, 10/2020}}
\end{center}
\end{titlepage}


    \cleardoublepage
    \footnotesize
    \pagenumbering{roman}
    \pagestyle{scrheadings}

    \normalsize
    %!TEX root = ../main.tex

%\renewcommand{\abstractname}{Kurzfassung}
\begin{abstract}
    % \addcontentsline{toc}{chapter}{Abstract}
    % \chapter*{Abstract}

    Human action recognition (HAR) has many implications in robotic and medical applications.
    Invariance under different viewpoints is one of the most critical requirements for practical deployment as it affects many aspects of the information captured such as occlusion, posture, color, shading, motion and background.
    In this thesis, a novel framework that leverages successful deep features for action representation and multi-view analysis to accomplish robust HAR under viewpoint changes.
    Specifically, various deep learning techniques, from 2D CNNs to 3D CNNs are investigated to capture spatial and temporal characteristics of actions at each individual view.
    A common feature space is then constructed to keep view invariant features among extracted streams.
    This is carried out by learning a set of linear transformations that projects separated view features into a common dimension.
    To this end, Multi-view Discriminant Analysis (MvDA) is adopted.
    However, the original MvDA suffers from odd situations in which the most class-discrepant common space could not be found because its objective is overly concentrated on scattering classes from the global mean but unaware of distance between specific pairs of classes.
    Therefore, we introduce a pairwise-covariance maximizing extension of MvDA that takes extra-class discriminance into account, namely pc-MvDA.
    The novel model also differs in the way that is more favorable for training of high-dimensional multi-view data.
    Experimental results on three datasets (IXMAS, MuHAVI, MICAGes) show the effectiveness of the proposed method.


\end{abstract}


% \begin{otherlanguage}{ngerman}
% %\renewcommand{\abstractname}{Kurzfassung}
% %\newcommand{\dtabstract}{\hyphenpenalty=10000}
% %{\dtabstract
% \begin{abstract}
% Das Gebiet des Music Information Retrieval befasst sich mit der automatischen Analyse von musikalischen Charakteristika. Ein Aspekt, der bisher kaum erforscht wurde, ist dabei der gesungene Text. Auf der anderen Seite werden in der automatischen Spracherkennung viele Methoden für die automatische Analyse von Sprache entwickelt, jedoch selten für Gesang. Die vorliegende Arbeit untersucht die Anwendung von Methoden aus der Spracherkennung auf Gesang und beschreibt mögliche Anpassungen. Zudem werden Wege zur praktischen Anwendung dieser Ansätze aufgezeigt. Fünf Themen werden dabei betrachtet: Phonemerkennung, Sprachenidentifikation, Schlagwortsuche, Text-zu-Gesangs-Alignment und Suche von Texten anhand von gesungenen Anfragen.\\
% Das größte Hindernis bei fast allen dieser Themen ist die Erkennung von Phonemen aus Gesangsaufnahmen. Herkömmliche, auf Sprache trainierte Modelle, bieten keine guten Ergebnisse für Gesang. Das Trainieren von Modellen auf Gesang ist schwierig, da kaum annotierte Daten verfügbar sind. Diese Arbeit zeigt zwei Ansätze auf, um solche Daten zu generieren. Für den ersten wurden Sprachaufnahmen künstlich gesangsähnlicher gemacht. Für den zweiten wurden Texte automatisch zu einem vorhandenen Gesangsdatensatz zugeordnet. Die neuen Datensätze wurden zum Trainieren neuer Modelle genutzt, welche deutliche Verbesserungen gegenüber sprachbasierten Modellen bieten.\\
% Auf diesen verbesserten akustischen Modellen aufbauend wurden Algorithmen aus der Spracherkennung für die verschiedenen Aufgaben angepasst, entweder durch das Verbessern der Robustheit gegenüber Gesangscharakteristika oder durch das Ausnutzen von hilfreichen Besonderheiten von Gesang. Beispiele für die verbesserte Robustheit sind der Einsatz von Keyword-Filler-HMMs für die Schlagwortsuche, ein i-Vector-Ansatz für die Sprachenidentifikation sowie eine Methode für das Alignment und die Textsuche, die stark schwankende Phonemdauern nicht bestraft. Die Besonderheiten von Gesang werden auf verschiedene Weisen genutzt: So z.B. in einem Ansatz für die Sprachenidentifikation, der lange Aufnahmen benötigt; in einer Methode für die Schlagwortsuche, die bekannte Phonemdauern in Gesang mit einbezieht; und in einem Algorithmus für das Alignment und die Textsuche, der bekannte Phonemkonfusionen verwertet.
% \end{abstract}
% \end{otherlanguage}

    %!TEX root = ../main.tex

\cleardoublepage
\ihead[]{Acknowledgements}
%\ohead[]{\pagemark}

% \addcontentsline{toc}{chapter}{Acknowledgements}
% \chapter*{Acknowledgements}

\begin{center}
\textbf{Acknowledgements}
\end{center}

This thesis would not have been possible without the help of many people.
First of all, I would like to express my gratitude to my primary advisor, Prof. Tran Thi Thanh Hai, who guided me throughout this project.
I would like to thank Prof. Le Thi Lan and Prof. Vu Hai for giving me deep insight, valuable recommendations and brilliant idea.

I am grateful for my time spent at MICA International Research Institute, where I learnt a lot about research and enjoyed a very warm and friendly working atmosphere.
In particular, I wish to extend my special thanks to PhD candidate. Nguyen Hong Quan and Dr. Doan Huong Giang who directly supported me.

Finally, I wish to show my appreciation to all my friends and family members who helped me finalizing the project.

\resethead


    \footnotesize
    \tableofcontents
    \normalsize

    \cleardoublepage
    \pagenumbering{arabic}
    \listoffigures
    \listoftables
    \listofabbreviations

    \cleardoublepage
    \input{sections/introduction}
    %!TEX root = ../../main.tex

\chapter{Technical Background and Related Works} \label{chap:background}
    \section{Introduction}
        This chapter provides the basic knowledge as well as related works regarding to the research topic of this thesis.
        Section \ref{sec:technical_background} introduces briefly the general architecture of deep neural networks for deep feature extraction; then describes in detail some dimensionality reduction algorithms and multi-view analysis algorithms.
        Section \ref{sec:related_works} summarizes approaches introduced in existing works to tackle problems in human action and gesture recognition and multi-view strategy.

    %!TEX root = ../../main.tex

\chapter{Technical Background and Related Works} \label{chap:background}
    \section{Introduction}
        This chapter provides the basic knowledge as well as related works regarding to the research topic of this thesis.
        Section \ref{sec:technical_background} introduces briefly the general architecture of deep neural networks for deep feature extraction; then describes in detail some dimensionality reduction algorithms and multi-view analysis algorithms.
        Section \ref{sec:related_works} summarizes approaches introduced in existing works to tackle problems in human action and gesture recognition and multi-view strategy.

    %!TEX root = ../../main.tex

\chapter{Technical Background and Related Works} \label{chap:background}
    \section{Introduction}
        This chapter provides the basic knowledge as well as related works regarding to the research topic of this thesis.
        Section \ref{sec:technical_background} introduces briefly the general architecture of deep neural networks for deep feature extraction; then describes in detail some dimensionality reduction algorithms and multi-view analysis algorithms.
        Section \ref{sec:related_works} summarizes approaches introduced in existing works to tackle problems in human action and gesture recognition and multi-view strategy.

    \input{sections/technical_background_and_related_works/technical_background/index}
    \input{sections/technical_background_and_related_works/related_works}

    \section{Summary}
        In this chapter, various related works were briefed.
        The general concepts of deep neural networks, especially 3D CNN because of its outstanding performance in action recognition for video data, are explained.
        Also, the underlying mathematical model behind LDA and two utilized multi-view analysis algorithms inspired by it (MvDA \& MvDA-vc) were clarified.
        In addition, an extension of LDA called pc-LDA that enhances it with better class discrepancy constraints were introduced.

    %!TEX root = ../../main.tex

\section{Related Works} \label{sec:related_works}

    \subsection{Human action and gesture recognition}
        Action recognition has been an attractive research topic since the last decade \cite{zhang2019comprehensive}.
        Early methods represented human actions by extracting 2D/3D key-points such as Harris-3D, SIFT-3D, HOG-3DHOF \cite{laptev2008learning}, ESURF \cite{willems2008efficient} then computed a descriptor from the detected key-points.
        Action representation  by a set of key-points could loose the temporal information. Therefore, Wang and Schmid in \cite{wang2013action} proposed a feature named improved dense trajectories (iDT) that densely sample and track optical flow points along trajectories.
        iDT has become state-of-the-art hand-crafted features and widely used for many video-based tasks.
        However, when working with large-scale datasets, iDT becomes intractable on due to its expensive computational cost and poor performance. 

        To work with more challenging datasets, effective action recognition approaches rely on powerful learning methods, particularly the deep learning techniques.
        Early works applied 2D CNN on frames of video sequence and then aggregated the information using pooling techniques \cite{karpathy2014large}.
        To exploit the temporal information, different architectures such as LSTM with the internal mechanisms called gates that can deal with short-term memory are proposed \cite{sun2017lattice}.
        Recently, instead of using 2D convolutional operators, different 3D CNN have been proposed \cite{ji20123d, tran2015learning, varol2017long}.
        Besides, to boost the recognition performance, different approaches tried to combine multiple streams \cite{wang2015towards, feichtenhofer2016convolutional, khong2018improving} or to combine both multiple features \cite{wang2015action, christoph2016spatiotemporal}. %trajectory-pooled deep-convolutional
        %descriptor - TDD;  ST-ResNet+iDT
        %Recently, many other architectures such as temporal Segment Networks - TSN \cite{wang2016temporal}, ST-VLMPF \cite{duta2017spatio}, P3D ResNet \cite{qiu2017learning}, I3D\cite{carreira2017quo}, 3D ResNeXt \cite{hara2018can}, R(2+1) D-TwoStream \cite{tran2018closer}, CO2FI+ASYN \cite{lin2018action}, and DML \cite{chen2017deep} have shown state-of-the-art performances in action recognition.
        %Previous section briefly gives a survey of different techniques for features extraction and action recognition from common views.

        These aforementioned approaches focus on single view action recognition, cross-view action recognition is more challenging and requires additional techniques to be taken into account. 
        Junejo et al. in \cite{junejo2008cross} proposed a descriptor, namely self-similarity matrix (SSM), which is an exhaustive table of distances between image features taken by pair from the image sequences.
        Liu et al. \cite{liu2011cross} employed cuboids extracted from each video and BoW model to build video descriptor for each single view. Then, a bipartite graph is built to model two view-dependent vocabularies.
        Li et al. \cite{li2012discriminative} described each video by concatenating spatio-temporal interest-point-based descriptor with shape flow descriptor. Then, to deal with cross-view, they construct ‘virtual views’, each is a linear transformation between action descriptors from one viewpoint and those from another.
        The method in \cite{zheng2012cross, zheng2013learning} employed the same video representation manner as in \cite{li2012discriminative}.
        However, a transferable dictionary between source and target view has been learnt to force features of the same action extracted from two views having the same sparse representation. 
        %In \cite{ulhaq2017space}, the authors proposed an advanced space-time filtering framework for recognizing human actions despite large viewpoint variations. Specifically, they used 3D tensor structure at each pixel, which characterizes the most common local motion in action sequences. Discrete tensor Fourier transform is then applied to achieve frequency domain representations. Then, they form view clusters from multiple-view action data and use space-time correlation filtering to achieve robust view representations. 

        Previous cross-view action recognition techniques usually connect source and target views with a set of linear transformations, that are unable to capture the non-linear manifolds on which real actions lie. In \cite{rahmani2017learning}, the authors find a shared high-level non-linear virtual path that connects multiple source and target views to the same canonical view. This virtual path is learnt by a deep neural network. In \cite{kong2017deeply}, a deep learning technique that stacks multiple layers of feature learners is designed to incorporate both private and shared view features. 
        In \cite{liu2018hierarchically}, the authors concatenated both private and shared view features and learnt transferable dictionary pair from a pair of views. In \cite{zhang2018action}, the authors proposed a framework to jointly learn a a view-invariant transfer dictionary and a view-invariant classifier using synthetic data during the pre-training phase to extract view-invariance between 3D and 2D videos.

        %As we have analyzed, multi-view  analysis has been actively studied. Several innovative ideas have been proposed. As a result, performance of multi-view classification is significantly improved. However, most of these methods are experimented on still images. The question of how still good those methods a special time-series data (i.e. video data) has been not addressed. In this paper, we contribute to improve the model of common feature space. In addition, we will evaluate the proposed method on video data, where both temporal and spatial features must be taken into account.

    \subsection{Multi-view analysis and learning techniques}
        As many objects in the real-world can be observed from different viewpoints, to exploit the consensual and complementary information between different views, Multi-view analysis (MvA) techniques are employed.
        MvA is a strategy for fusing data from different sources or subsets.

        Canonical Correlation Analysis (CCA) \cite{Hotelling} can be considered as the first approach of MvL with the aim to find pairs of projections for two views so that the correlations between these views are maximized.
        As CCA can only handle the linear correlation, Kernel CCA (KCCA) was proposed to take non-linear correlation relationship of data into account \cite{Akaho2006}.
        However, both CCA and KCCA are unsupervised methods and can not leverage the label information.
        In \cite{diethe2008multiview}, a supervised approach named Multi-view Fisher discriminant analysis (MvFDA) was proposed for binary classification problem.
        All of aforementioned methods are only applicable for two views problem.

        To extend to multiple view cases, a natural extension is to maximize the sum of the pairwise correlations.
        In a general case, it would be better to build a common shared feature space that captures latent information of the object from all observed views.
        For this propose, Multi-view CCA (MvCCA) is proposed in 2010 to build a common feature space of all views \cite{rupnik2010multi}. 
        %MCCA tried to find $v$ transformations by maximizing the correlation of every two views. 
        However, MvCCA did not consider the discrepancy information but only maximizing the correlation between every two views, so that it may be ineffective for classification across views. 
        %Generalized Multi-view Analysis (GMA) preserves the supervised structure of each view while keeping the projections of different views close to each other in the latent common space. GMA is considered as an extension of Fisher Discriminant analysis (FDA) for cross-view problem. It considers class label information so it could be good for multi-view classification. However, GMA considers only the the discriminant information in each individual view, not inter-view so it could decrease cross-view recognition.
        %Multi-view Uncorrelated Linear Discriminant Analysis (MULDA) \cite{sun2015multi-view} learnt uncorrelated discriminant features by using Uncorrelated Linear Discriminant Analysis (ULDA). Multi-view Modular Discriminant Analysis (MvMDA) \cite{cao2017generalized} was proposed to separate class centers across different views. 

        In \cite{kan2015multi}, Multi-view discriminant analysis (MvDA), an extension of linear discriminant analysis (LDA) for multi-view problem was proposed.
        MvDA tries to optimize jointly view correlation, intra-view and inter-view discriminability. 
        An extension of MvDA which considers view-consistency was also introduced and achieved significant performance improvement.

        In \cite{zhao2018multi}, the authors proposed multi-view manifold learning with locality alignment (MvML-LA) framework to realize manifold learning under multi-view scenario. 
        %Locality alignment in the latent space learning is considered to enhance its discriminative capability and developed two specific algorithms in supervised and unsupervised scenarios, respectively. 

        Most recently, \cite{you2019multi} proposed Multi-view Common Component Discriminant Analysis (MvCCDA) technique that both integrates supervised information and local geometric information into the common component extraction process.
        This helps to effectively handle view discrepancy, discriminability and non-linearity in a joint manner.


    \section{Summary}
        In this chapter, various related works were briefed.
        The general concepts of deep neural networks, especially 3D CNN because of its outstanding performance in action recognition for video data, are explained.
        Also, the underlying mathematical model behind LDA and two utilized multi-view analysis algorithms inspired by it (MvDA \& MvDA-vc) were clarified.
        In addition, an extension of LDA called pc-LDA that enhances it with better class discrepancy constraints were introduced.

    %!TEX root = ../../main.tex

\section{Related Works} \label{sec:related_works}

    \subsection{Human action and gesture recognition}
        Action recognition has been an attractive research topic since the last decade \cite{zhang2019comprehensive}.
        Early methods represented human actions by extracting 2D/3D key-points such as Harris-3D, SIFT-3D, HOG-3DHOF \cite{laptev2008learning}, ESURF \cite{willems2008efficient} then computed a descriptor from the detected key-points.
        Action representation  by a set of key-points could loose the temporal information. Therefore, Wang and Schmid in \cite{wang2013action} proposed a feature named improved dense trajectories (iDT) that densely sample and track optical flow points along trajectories.
        iDT has become state-of-the-art hand-crafted features and widely used for many video-based tasks.
        However, when working with large-scale datasets, iDT becomes intractable on due to its expensive computational cost and poor performance. 

        To work with more challenging datasets, effective action recognition approaches rely on powerful learning methods, particularly the deep learning techniques.
        Early works applied 2D CNN on frames of video sequence and then aggregated the information using pooling techniques \cite{karpathy2014large}.
        To exploit the temporal information, different architectures such as LSTM with the internal mechanisms called gates that can deal with short-term memory are proposed \cite{sun2017lattice}.
        Recently, instead of using 2D convolutional operators, different 3D CNN have been proposed \cite{ji20123d, tran2015learning, varol2017long}.
        Besides, to boost the recognition performance, different approaches tried to combine multiple streams \cite{wang2015towards, feichtenhofer2016convolutional, khong2018improving} or to combine both multiple features \cite{wang2015action, christoph2016spatiotemporal}. %trajectory-pooled deep-convolutional
        %descriptor - TDD;  ST-ResNet+iDT
        %Recently, many other architectures such as temporal Segment Networks - TSN \cite{wang2016temporal}, ST-VLMPF \cite{duta2017spatio}, P3D ResNet \cite{qiu2017learning}, I3D\cite{carreira2017quo}, 3D ResNeXt \cite{hara2018can}, R(2+1) D-TwoStream \cite{tran2018closer}, CO2FI+ASYN \cite{lin2018action}, and DML \cite{chen2017deep} have shown state-of-the-art performances in action recognition.
        %Previous section briefly gives a survey of different techniques for features extraction and action recognition from common views.

        These aforementioned approaches focus on single view action recognition, cross-view action recognition is more challenging and requires additional techniques to be taken into account. 
        Junejo et al. in \cite{junejo2008cross} proposed a descriptor, namely self-similarity matrix (SSM), which is an exhaustive table of distances between image features taken by pair from the image sequences.
        Liu et al. \cite{liu2011cross} employed cuboids extracted from each video and BoW model to build video descriptor for each single view. Then, a bipartite graph is built to model two view-dependent vocabularies.
        Li et al. \cite{li2012discriminative} described each video by concatenating spatio-temporal interest-point-based descriptor with shape flow descriptor. Then, to deal with cross-view, they construct ‘virtual views’, each is a linear transformation between action descriptors from one viewpoint and those from another.
        The method in \cite{zheng2012cross, zheng2013learning} employed the same video representation manner as in \cite{li2012discriminative}.
        However, a transferable dictionary between source and target view has been learnt to force features of the same action extracted from two views having the same sparse representation. 
        %In \cite{ulhaq2017space}, the authors proposed an advanced space-time filtering framework for recognizing human actions despite large viewpoint variations. Specifically, they used 3D tensor structure at each pixel, which characterizes the most common local motion in action sequences. Discrete tensor Fourier transform is then applied to achieve frequency domain representations. Then, they form view clusters from multiple-view action data and use space-time correlation filtering to achieve robust view representations. 

        Previous cross-view action recognition techniques usually connect source and target views with a set of linear transformations, that are unable to capture the non-linear manifolds on which real actions lie. In \cite{rahmani2017learning}, the authors find a shared high-level non-linear virtual path that connects multiple source and target views to the same canonical view. This virtual path is learnt by a deep neural network. In \cite{kong2017deeply}, a deep learning technique that stacks multiple layers of feature learners is designed to incorporate both private and shared view features. 
        In \cite{liu2018hierarchically}, the authors concatenated both private and shared view features and learnt transferable dictionary pair from a pair of views. In \cite{zhang2018action}, the authors proposed a framework to jointly learn a a view-invariant transfer dictionary and a view-invariant classifier using synthetic data during the pre-training phase to extract view-invariance between 3D and 2D videos.

        %As we have analyzed, multi-view  analysis has been actively studied. Several innovative ideas have been proposed. As a result, performance of multi-view classification is significantly improved. However, most of these methods are experimented on still images. The question of how still good those methods a special time-series data (i.e. video data) has been not addressed. In this paper, we contribute to improve the model of common feature space. In addition, we will evaluate the proposed method on video data, where both temporal and spatial features must be taken into account.

    \subsection{Multi-view analysis and learning techniques}
        As many objects in the real-world can be observed from different viewpoints, to exploit the consensual and complementary information between different views, Multi-view analysis (MvA) techniques are employed.
        MvA is a strategy for fusing data from different sources or subsets.

        Canonical Correlation Analysis (CCA) \cite{Hotelling} can be considered as the first approach of MvL with the aim to find pairs of projections for two views so that the correlations between these views are maximized.
        As CCA can only handle the linear correlation, Kernel CCA (KCCA) was proposed to take non-linear correlation relationship of data into account \cite{Akaho2006}.
        However, both CCA and KCCA are unsupervised methods and can not leverage the label information.
        In \cite{diethe2008multiview}, a supervised approach named Multi-view Fisher discriminant analysis (MvFDA) was proposed for binary classification problem.
        All of aforementioned methods are only applicable for two views problem.

        To extend to multiple view cases, a natural extension is to maximize the sum of the pairwise correlations.
        In a general case, it would be better to build a common shared feature space that captures latent information of the object from all observed views.
        For this propose, Multi-view CCA (MvCCA) is proposed in 2010 to build a common feature space of all views \cite{rupnik2010multi}. 
        %MCCA tried to find $v$ transformations by maximizing the correlation of every two views. 
        However, MvCCA did not consider the discrepancy information but only maximizing the correlation between every two views, so that it may be ineffective for classification across views. 
        %Generalized Multi-view Analysis (GMA) preserves the supervised structure of each view while keeping the projections of different views close to each other in the latent common space. GMA is considered as an extension of Fisher Discriminant analysis (FDA) for cross-view problem. It considers class label information so it could be good for multi-view classification. However, GMA considers only the the discriminant information in each individual view, not inter-view so it could decrease cross-view recognition.
        %Multi-view Uncorrelated Linear Discriminant Analysis (MULDA) \cite{sun2015multi-view} learnt uncorrelated discriminant features by using Uncorrelated Linear Discriminant Analysis (ULDA). Multi-view Modular Discriminant Analysis (MvMDA) \cite{cao2017generalized} was proposed to separate class centers across different views. 

        In \cite{kan2015multi}, Multi-view discriminant analysis (MvDA), an extension of linear discriminant analysis (LDA) for multi-view problem was proposed.
        MvDA tries to optimize jointly view correlation, intra-view and inter-view discriminability. 
        An extension of MvDA which considers view-consistency was also introduced and achieved significant performance improvement.

        In \cite{zhao2018multi}, the authors proposed multi-view manifold learning with locality alignment (MvML-LA) framework to realize manifold learning under multi-view scenario. 
        %Locality alignment in the latent space learning is considered to enhance its discriminative capability and developed two specific algorithms in supervised and unsupervised scenarios, respectively. 

        Most recently, \cite{you2019multi} proposed Multi-view Common Component Discriminant Analysis (MvCCDA) technique that both integrates supervised information and local geometric information into the common component extraction process.
        This helps to effectively handle view discrepancy, discriminability and non-linearity in a joint manner.


    \section{Summary}
        In this chapter, various related works were briefed.
        The general concepts of deep neural networks, especially 3D CNN because of its outstanding performance in action recognition for video data, are explained.
        Also, the underlying mathematical model behind LDA and two utilized multi-view analysis algorithms inspired by it (MvDA \& MvDA-vc) were clarified.
        In addition, an extension of LDA called pc-LDA that enhances it with better class discrepancy constraints were introduced.

    %!TEX root = ../../main.tex

\chapter{Technical Background and Related Works} \label{chap:background}
    \section{Introduction}
        This chapter provides the basic knowledge as well as related works regarding to the research topic of this thesis.
        Section \ref{sec:technical_background} introduces briefly the general architecture of deep neural networks for deep feature extraction; then describes in detail some dimensionality reduction algorithms and multi-view analysis algorithms.
        Section \ref{sec:related_works} summarizes approaches introduced in existing works to tackle problems in human action and gesture recognition and multi-view strategy.

    %!TEX root = ../../main.tex

\chapter{Technical Background and Related Works} \label{chap:background}
    \section{Introduction}
        This chapter provides the basic knowledge as well as related works regarding to the research topic of this thesis.
        Section \ref{sec:technical_background} introduces briefly the general architecture of deep neural networks for deep feature extraction; then describes in detail some dimensionality reduction algorithms and multi-view analysis algorithms.
        Section \ref{sec:related_works} summarizes approaches introduced in existing works to tackle problems in human action and gesture recognition and multi-view strategy.

    %!TEX root = ../../main.tex

\chapter{Technical Background and Related Works} \label{chap:background}
    \section{Introduction}
        This chapter provides the basic knowledge as well as related works regarding to the research topic of this thesis.
        Section \ref{sec:technical_background} introduces briefly the general architecture of deep neural networks for deep feature extraction; then describes in detail some dimensionality reduction algorithms and multi-view analysis algorithms.
        Section \ref{sec:related_works} summarizes approaches introduced in existing works to tackle problems in human action and gesture recognition and multi-view strategy.

    \input{sections/technical_background_and_related_works/technical_background/index}
    \input{sections/technical_background_and_related_works/related_works}

    \section{Summary}
        In this chapter, various related works were briefed.
        The general concepts of deep neural networks, especially 3D CNN because of its outstanding performance in action recognition for video data, are explained.
        Also, the underlying mathematical model behind LDA and two utilized multi-view analysis algorithms inspired by it (MvDA \& MvDA-vc) were clarified.
        In addition, an extension of LDA called pc-LDA that enhances it with better class discrepancy constraints were introduced.

    %!TEX root = ../../main.tex

\section{Related Works} \label{sec:related_works}

    \subsection{Human action and gesture recognition}
        Action recognition has been an attractive research topic since the last decade \cite{zhang2019comprehensive}.
        Early methods represented human actions by extracting 2D/3D key-points such as Harris-3D, SIFT-3D, HOG-3DHOF \cite{laptev2008learning}, ESURF \cite{willems2008efficient} then computed a descriptor from the detected key-points.
        Action representation  by a set of key-points could loose the temporal information. Therefore, Wang and Schmid in \cite{wang2013action} proposed a feature named improved dense trajectories (iDT) that densely sample and track optical flow points along trajectories.
        iDT has become state-of-the-art hand-crafted features and widely used for many video-based tasks.
        However, when working with large-scale datasets, iDT becomes intractable on due to its expensive computational cost and poor performance. 

        To work with more challenging datasets, effective action recognition approaches rely on powerful learning methods, particularly the deep learning techniques.
        Early works applied 2D CNN on frames of video sequence and then aggregated the information using pooling techniques \cite{karpathy2014large}.
        To exploit the temporal information, different architectures such as LSTM with the internal mechanisms called gates that can deal with short-term memory are proposed \cite{sun2017lattice}.
        Recently, instead of using 2D convolutional operators, different 3D CNN have been proposed \cite{ji20123d, tran2015learning, varol2017long}.
        Besides, to boost the recognition performance, different approaches tried to combine multiple streams \cite{wang2015towards, feichtenhofer2016convolutional, khong2018improving} or to combine both multiple features \cite{wang2015action, christoph2016spatiotemporal}. %trajectory-pooled deep-convolutional
        %descriptor - TDD;  ST-ResNet+iDT
        %Recently, many other architectures such as temporal Segment Networks - TSN \cite{wang2016temporal}, ST-VLMPF \cite{duta2017spatio}, P3D ResNet \cite{qiu2017learning}, I3D\cite{carreira2017quo}, 3D ResNeXt \cite{hara2018can}, R(2+1) D-TwoStream \cite{tran2018closer}, CO2FI+ASYN \cite{lin2018action}, and DML \cite{chen2017deep} have shown state-of-the-art performances in action recognition.
        %Previous section briefly gives a survey of different techniques for features extraction and action recognition from common views.

        These aforementioned approaches focus on single view action recognition, cross-view action recognition is more challenging and requires additional techniques to be taken into account. 
        Junejo et al. in \cite{junejo2008cross} proposed a descriptor, namely self-similarity matrix (SSM), which is an exhaustive table of distances between image features taken by pair from the image sequences.
        Liu et al. \cite{liu2011cross} employed cuboids extracted from each video and BoW model to build video descriptor for each single view. Then, a bipartite graph is built to model two view-dependent vocabularies.
        Li et al. \cite{li2012discriminative} described each video by concatenating spatio-temporal interest-point-based descriptor with shape flow descriptor. Then, to deal with cross-view, they construct ‘virtual views’, each is a linear transformation between action descriptors from one viewpoint and those from another.
        The method in \cite{zheng2012cross, zheng2013learning} employed the same video representation manner as in \cite{li2012discriminative}.
        However, a transferable dictionary between source and target view has been learnt to force features of the same action extracted from two views having the same sparse representation. 
        %In \cite{ulhaq2017space}, the authors proposed an advanced space-time filtering framework for recognizing human actions despite large viewpoint variations. Specifically, they used 3D tensor structure at each pixel, which characterizes the most common local motion in action sequences. Discrete tensor Fourier transform is then applied to achieve frequency domain representations. Then, they form view clusters from multiple-view action data and use space-time correlation filtering to achieve robust view representations. 

        Previous cross-view action recognition techniques usually connect source and target views with a set of linear transformations, that are unable to capture the non-linear manifolds on which real actions lie. In \cite{rahmani2017learning}, the authors find a shared high-level non-linear virtual path that connects multiple source and target views to the same canonical view. This virtual path is learnt by a deep neural network. In \cite{kong2017deeply}, a deep learning technique that stacks multiple layers of feature learners is designed to incorporate both private and shared view features. 
        In \cite{liu2018hierarchically}, the authors concatenated both private and shared view features and learnt transferable dictionary pair from a pair of views. In \cite{zhang2018action}, the authors proposed a framework to jointly learn a a view-invariant transfer dictionary and a view-invariant classifier using synthetic data during the pre-training phase to extract view-invariance between 3D and 2D videos.

        %As we have analyzed, multi-view  analysis has been actively studied. Several innovative ideas have been proposed. As a result, performance of multi-view classification is significantly improved. However, most of these methods are experimented on still images. The question of how still good those methods a special time-series data (i.e. video data) has been not addressed. In this paper, we contribute to improve the model of common feature space. In addition, we will evaluate the proposed method on video data, where both temporal and spatial features must be taken into account.

    \subsection{Multi-view analysis and learning techniques}
        As many objects in the real-world can be observed from different viewpoints, to exploit the consensual and complementary information between different views, Multi-view analysis (MvA) techniques are employed.
        MvA is a strategy for fusing data from different sources or subsets.

        Canonical Correlation Analysis (CCA) \cite{Hotelling} can be considered as the first approach of MvL with the aim to find pairs of projections for two views so that the correlations between these views are maximized.
        As CCA can only handle the linear correlation, Kernel CCA (KCCA) was proposed to take non-linear correlation relationship of data into account \cite{Akaho2006}.
        However, both CCA and KCCA are unsupervised methods and can not leverage the label information.
        In \cite{diethe2008multiview}, a supervised approach named Multi-view Fisher discriminant analysis (MvFDA) was proposed for binary classification problem.
        All of aforementioned methods are only applicable for two views problem.

        To extend to multiple view cases, a natural extension is to maximize the sum of the pairwise correlations.
        In a general case, it would be better to build a common shared feature space that captures latent information of the object from all observed views.
        For this propose, Multi-view CCA (MvCCA) is proposed in 2010 to build a common feature space of all views \cite{rupnik2010multi}. 
        %MCCA tried to find $v$ transformations by maximizing the correlation of every two views. 
        However, MvCCA did not consider the discrepancy information but only maximizing the correlation between every two views, so that it may be ineffective for classification across views. 
        %Generalized Multi-view Analysis (GMA) preserves the supervised structure of each view while keeping the projections of different views close to each other in the latent common space. GMA is considered as an extension of Fisher Discriminant analysis (FDA) for cross-view problem. It considers class label information so it could be good for multi-view classification. However, GMA considers only the the discriminant information in each individual view, not inter-view so it could decrease cross-view recognition.
        %Multi-view Uncorrelated Linear Discriminant Analysis (MULDA) \cite{sun2015multi-view} learnt uncorrelated discriminant features by using Uncorrelated Linear Discriminant Analysis (ULDA). Multi-view Modular Discriminant Analysis (MvMDA) \cite{cao2017generalized} was proposed to separate class centers across different views. 

        In \cite{kan2015multi}, Multi-view discriminant analysis (MvDA), an extension of linear discriminant analysis (LDA) for multi-view problem was proposed.
        MvDA tries to optimize jointly view correlation, intra-view and inter-view discriminability. 
        An extension of MvDA which considers view-consistency was also introduced and achieved significant performance improvement.

        In \cite{zhao2018multi}, the authors proposed multi-view manifold learning with locality alignment (MvML-LA) framework to realize manifold learning under multi-view scenario. 
        %Locality alignment in the latent space learning is considered to enhance its discriminative capability and developed two specific algorithms in supervised and unsupervised scenarios, respectively. 

        Most recently, \cite{you2019multi} proposed Multi-view Common Component Discriminant Analysis (MvCCDA) technique that both integrates supervised information and local geometric information into the common component extraction process.
        This helps to effectively handle view discrepancy, discriminability and non-linearity in a joint manner.


    \section{Summary}
        In this chapter, various related works were briefed.
        The general concepts of deep neural networks, especially 3D CNN because of its outstanding performance in action recognition for video data, are explained.
        Also, the underlying mathematical model behind LDA and two utilized multi-view analysis algorithms inspired by it (MvDA \& MvDA-vc) were clarified.
        In addition, an extension of LDA called pc-LDA that enhances it with better class discrepancy constraints were introduced.

    %!TEX root = ../../main.tex

\section{Related Works} \label{sec:related_works}

    \subsection{Human action and gesture recognition}
        Action recognition has been an attractive research topic since the last decade \cite{zhang2019comprehensive}.
        Early methods represented human actions by extracting 2D/3D key-points such as Harris-3D, SIFT-3D, HOG-3DHOF \cite{laptev2008learning}, ESURF \cite{willems2008efficient} then computed a descriptor from the detected key-points.
        Action representation  by a set of key-points could loose the temporal information. Therefore, Wang and Schmid in \cite{wang2013action} proposed a feature named improved dense trajectories (iDT) that densely sample and track optical flow points along trajectories.
        iDT has become state-of-the-art hand-crafted features and widely used for many video-based tasks.
        However, when working with large-scale datasets, iDT becomes intractable on due to its expensive computational cost and poor performance. 

        To work with more challenging datasets, effective action recognition approaches rely on powerful learning methods, particularly the deep learning techniques.
        Early works applied 2D CNN on frames of video sequence and then aggregated the information using pooling techniques \cite{karpathy2014large}.
        To exploit the temporal information, different architectures such as LSTM with the internal mechanisms called gates that can deal with short-term memory are proposed \cite{sun2017lattice}.
        Recently, instead of using 2D convolutional operators, different 3D CNN have been proposed \cite{ji20123d, tran2015learning, varol2017long}.
        Besides, to boost the recognition performance, different approaches tried to combine multiple streams \cite{wang2015towards, feichtenhofer2016convolutional, khong2018improving} or to combine both multiple features \cite{wang2015action, christoph2016spatiotemporal}. %trajectory-pooled deep-convolutional
        %descriptor - TDD;  ST-ResNet+iDT
        %Recently, many other architectures such as temporal Segment Networks - TSN \cite{wang2016temporal}, ST-VLMPF \cite{duta2017spatio}, P3D ResNet \cite{qiu2017learning}, I3D\cite{carreira2017quo}, 3D ResNeXt \cite{hara2018can}, R(2+1) D-TwoStream \cite{tran2018closer}, CO2FI+ASYN \cite{lin2018action}, and DML \cite{chen2017deep} have shown state-of-the-art performances in action recognition.
        %Previous section briefly gives a survey of different techniques for features extraction and action recognition from common views.

        These aforementioned approaches focus on single view action recognition, cross-view action recognition is more challenging and requires additional techniques to be taken into account. 
        Junejo et al. in \cite{junejo2008cross} proposed a descriptor, namely self-similarity matrix (SSM), which is an exhaustive table of distances between image features taken by pair from the image sequences.
        Liu et al. \cite{liu2011cross} employed cuboids extracted from each video and BoW model to build video descriptor for each single view. Then, a bipartite graph is built to model two view-dependent vocabularies.
        Li et al. \cite{li2012discriminative} described each video by concatenating spatio-temporal interest-point-based descriptor with shape flow descriptor. Then, to deal with cross-view, they construct ‘virtual views’, each is a linear transformation between action descriptors from one viewpoint and those from another.
        The method in \cite{zheng2012cross, zheng2013learning} employed the same video representation manner as in \cite{li2012discriminative}.
        However, a transferable dictionary between source and target view has been learnt to force features of the same action extracted from two views having the same sparse representation. 
        %In \cite{ulhaq2017space}, the authors proposed an advanced space-time filtering framework for recognizing human actions despite large viewpoint variations. Specifically, they used 3D tensor structure at each pixel, which characterizes the most common local motion in action sequences. Discrete tensor Fourier transform is then applied to achieve frequency domain representations. Then, they form view clusters from multiple-view action data and use space-time correlation filtering to achieve robust view representations. 

        Previous cross-view action recognition techniques usually connect source and target views with a set of linear transformations, that are unable to capture the non-linear manifolds on which real actions lie. In \cite{rahmani2017learning}, the authors find a shared high-level non-linear virtual path that connects multiple source and target views to the same canonical view. This virtual path is learnt by a deep neural network. In \cite{kong2017deeply}, a deep learning technique that stacks multiple layers of feature learners is designed to incorporate both private and shared view features. 
        In \cite{liu2018hierarchically}, the authors concatenated both private and shared view features and learnt transferable dictionary pair from a pair of views. In \cite{zhang2018action}, the authors proposed a framework to jointly learn a a view-invariant transfer dictionary and a view-invariant classifier using synthetic data during the pre-training phase to extract view-invariance between 3D and 2D videos.

        %As we have analyzed, multi-view  analysis has been actively studied. Several innovative ideas have been proposed. As a result, performance of multi-view classification is significantly improved. However, most of these methods are experimented on still images. The question of how still good those methods a special time-series data (i.e. video data) has been not addressed. In this paper, we contribute to improve the model of common feature space. In addition, we will evaluate the proposed method on video data, where both temporal and spatial features must be taken into account.

    \subsection{Multi-view analysis and learning techniques}
        As many objects in the real-world can be observed from different viewpoints, to exploit the consensual and complementary information between different views, Multi-view analysis (MvA) techniques are employed.
        MvA is a strategy for fusing data from different sources or subsets.

        Canonical Correlation Analysis (CCA) \cite{Hotelling} can be considered as the first approach of MvL with the aim to find pairs of projections for two views so that the correlations between these views are maximized.
        As CCA can only handle the linear correlation, Kernel CCA (KCCA) was proposed to take non-linear correlation relationship of data into account \cite{Akaho2006}.
        However, both CCA and KCCA are unsupervised methods and can not leverage the label information.
        In \cite{diethe2008multiview}, a supervised approach named Multi-view Fisher discriminant analysis (MvFDA) was proposed for binary classification problem.
        All of aforementioned methods are only applicable for two views problem.

        To extend to multiple view cases, a natural extension is to maximize the sum of the pairwise correlations.
        In a general case, it would be better to build a common shared feature space that captures latent information of the object from all observed views.
        For this propose, Multi-view CCA (MvCCA) is proposed in 2010 to build a common feature space of all views \cite{rupnik2010multi}. 
        %MCCA tried to find $v$ transformations by maximizing the correlation of every two views. 
        However, MvCCA did not consider the discrepancy information but only maximizing the correlation between every two views, so that it may be ineffective for classification across views. 
        %Generalized Multi-view Analysis (GMA) preserves the supervised structure of each view while keeping the projections of different views close to each other in the latent common space. GMA is considered as an extension of Fisher Discriminant analysis (FDA) for cross-view problem. It considers class label information so it could be good for multi-view classification. However, GMA considers only the the discriminant information in each individual view, not inter-view so it could decrease cross-view recognition.
        %Multi-view Uncorrelated Linear Discriminant Analysis (MULDA) \cite{sun2015multi-view} learnt uncorrelated discriminant features by using Uncorrelated Linear Discriminant Analysis (ULDA). Multi-view Modular Discriminant Analysis (MvMDA) \cite{cao2017generalized} was proposed to separate class centers across different views. 

        In \cite{kan2015multi}, Multi-view discriminant analysis (MvDA), an extension of linear discriminant analysis (LDA) for multi-view problem was proposed.
        MvDA tries to optimize jointly view correlation, intra-view and inter-view discriminability. 
        An extension of MvDA which considers view-consistency was also introduced and achieved significant performance improvement.

        In \cite{zhao2018multi}, the authors proposed multi-view manifold learning with locality alignment (MvML-LA) framework to realize manifold learning under multi-view scenario. 
        %Locality alignment in the latent space learning is considered to enhance its discriminative capability and developed two specific algorithms in supervised and unsupervised scenarios, respectively. 

        Most recently, \cite{you2019multi} proposed Multi-view Common Component Discriminant Analysis (MvCCDA) technique that both integrates supervised information and local geometric information into the common component extraction process.
        This helps to effectively handle view discrepancy, discriminability and non-linearity in a joint manner.


    \section{Summary}
        In this chapter, various related works were briefed.
        The general concepts of deep neural networks, especially 3D CNN because of its outstanding performance in action recognition for video data, are explained.
        Also, the underlying mathematical model behind LDA and two utilized multi-view analysis algorithms inspired by it (MvDA \& MvDA-vc) were clarified.
        In addition, an extension of LDA called pc-LDA that enhances it with better class discrepancy constraints were introduced.

    %!TEX root = ../../main.tex

\chapter{Technical Background and Related Works} \label{chap:background}
    \section{Introduction}
        This chapter provides the basic knowledge as well as related works regarding to the research topic of this thesis.
        Section \ref{sec:technical_background} introduces briefly the general architecture of deep neural networks for deep feature extraction; then describes in detail some dimensionality reduction algorithms and multi-view analysis algorithms.
        Section \ref{sec:related_works} summarizes approaches introduced in existing works to tackle problems in human action and gesture recognition and multi-view strategy.

    %!TEX root = ../../main.tex

\chapter{Technical Background and Related Works} \label{chap:background}
    \section{Introduction}
        This chapter provides the basic knowledge as well as related works regarding to the research topic of this thesis.
        Section \ref{sec:technical_background} introduces briefly the general architecture of deep neural networks for deep feature extraction; then describes in detail some dimensionality reduction algorithms and multi-view analysis algorithms.
        Section \ref{sec:related_works} summarizes approaches introduced in existing works to tackle problems in human action and gesture recognition and multi-view strategy.

    %!TEX root = ../../main.tex

\chapter{Technical Background and Related Works} \label{chap:background}
    \section{Introduction}
        This chapter provides the basic knowledge as well as related works regarding to the research topic of this thesis.
        Section \ref{sec:technical_background} introduces briefly the general architecture of deep neural networks for deep feature extraction; then describes in detail some dimensionality reduction algorithms and multi-view analysis algorithms.
        Section \ref{sec:related_works} summarizes approaches introduced in existing works to tackle problems in human action and gesture recognition and multi-view strategy.

    \input{sections/technical_background_and_related_works/technical_background/index}
    \input{sections/technical_background_and_related_works/related_works}

    \section{Summary}
        In this chapter, various related works were briefed.
        The general concepts of deep neural networks, especially 3D CNN because of its outstanding performance in action recognition for video data, are explained.
        Also, the underlying mathematical model behind LDA and two utilized multi-view analysis algorithms inspired by it (MvDA \& MvDA-vc) were clarified.
        In addition, an extension of LDA called pc-LDA that enhances it with better class discrepancy constraints were introduced.

    %!TEX root = ../../main.tex

\section{Related Works} \label{sec:related_works}

    \subsection{Human action and gesture recognition}
        Action recognition has been an attractive research topic since the last decade \cite{zhang2019comprehensive}.
        Early methods represented human actions by extracting 2D/3D key-points such as Harris-3D, SIFT-3D, HOG-3DHOF \cite{laptev2008learning}, ESURF \cite{willems2008efficient} then computed a descriptor from the detected key-points.
        Action representation  by a set of key-points could loose the temporal information. Therefore, Wang and Schmid in \cite{wang2013action} proposed a feature named improved dense trajectories (iDT) that densely sample and track optical flow points along trajectories.
        iDT has become state-of-the-art hand-crafted features and widely used for many video-based tasks.
        However, when working with large-scale datasets, iDT becomes intractable on due to its expensive computational cost and poor performance. 

        To work with more challenging datasets, effective action recognition approaches rely on powerful learning methods, particularly the deep learning techniques.
        Early works applied 2D CNN on frames of video sequence and then aggregated the information using pooling techniques \cite{karpathy2014large}.
        To exploit the temporal information, different architectures such as LSTM with the internal mechanisms called gates that can deal with short-term memory are proposed \cite{sun2017lattice}.
        Recently, instead of using 2D convolutional operators, different 3D CNN have been proposed \cite{ji20123d, tran2015learning, varol2017long}.
        Besides, to boost the recognition performance, different approaches tried to combine multiple streams \cite{wang2015towards, feichtenhofer2016convolutional, khong2018improving} or to combine both multiple features \cite{wang2015action, christoph2016spatiotemporal}. %trajectory-pooled deep-convolutional
        %descriptor - TDD;  ST-ResNet+iDT
        %Recently, many other architectures such as temporal Segment Networks - TSN \cite{wang2016temporal}, ST-VLMPF \cite{duta2017spatio}, P3D ResNet \cite{qiu2017learning}, I3D\cite{carreira2017quo}, 3D ResNeXt \cite{hara2018can}, R(2+1) D-TwoStream \cite{tran2018closer}, CO2FI+ASYN \cite{lin2018action}, and DML \cite{chen2017deep} have shown state-of-the-art performances in action recognition.
        %Previous section briefly gives a survey of different techniques for features extraction and action recognition from common views.

        These aforementioned approaches focus on single view action recognition, cross-view action recognition is more challenging and requires additional techniques to be taken into account. 
        Junejo et al. in \cite{junejo2008cross} proposed a descriptor, namely self-similarity matrix (SSM), which is an exhaustive table of distances between image features taken by pair from the image sequences.
        Liu et al. \cite{liu2011cross} employed cuboids extracted from each video and BoW model to build video descriptor for each single view. Then, a bipartite graph is built to model two view-dependent vocabularies.
        Li et al. \cite{li2012discriminative} described each video by concatenating spatio-temporal interest-point-based descriptor with shape flow descriptor. Then, to deal with cross-view, they construct ‘virtual views’, each is a linear transformation between action descriptors from one viewpoint and those from another.
        The method in \cite{zheng2012cross, zheng2013learning} employed the same video representation manner as in \cite{li2012discriminative}.
        However, a transferable dictionary between source and target view has been learnt to force features of the same action extracted from two views having the same sparse representation. 
        %In \cite{ulhaq2017space}, the authors proposed an advanced space-time filtering framework for recognizing human actions despite large viewpoint variations. Specifically, they used 3D tensor structure at each pixel, which characterizes the most common local motion in action sequences. Discrete tensor Fourier transform is then applied to achieve frequency domain representations. Then, they form view clusters from multiple-view action data and use space-time correlation filtering to achieve robust view representations. 

        Previous cross-view action recognition techniques usually connect source and target views with a set of linear transformations, that are unable to capture the non-linear manifolds on which real actions lie. In \cite{rahmani2017learning}, the authors find a shared high-level non-linear virtual path that connects multiple source and target views to the same canonical view. This virtual path is learnt by a deep neural network. In \cite{kong2017deeply}, a deep learning technique that stacks multiple layers of feature learners is designed to incorporate both private and shared view features. 
        In \cite{liu2018hierarchically}, the authors concatenated both private and shared view features and learnt transferable dictionary pair from a pair of views. In \cite{zhang2018action}, the authors proposed a framework to jointly learn a a view-invariant transfer dictionary and a view-invariant classifier using synthetic data during the pre-training phase to extract view-invariance between 3D and 2D videos.

        %As we have analyzed, multi-view  analysis has been actively studied. Several innovative ideas have been proposed. As a result, performance of multi-view classification is significantly improved. However, most of these methods are experimented on still images. The question of how still good those methods a special time-series data (i.e. video data) has been not addressed. In this paper, we contribute to improve the model of common feature space. In addition, we will evaluate the proposed method on video data, where both temporal and spatial features must be taken into account.

    \subsection{Multi-view analysis and learning techniques}
        As many objects in the real-world can be observed from different viewpoints, to exploit the consensual and complementary information between different views, Multi-view analysis (MvA) techniques are employed.
        MvA is a strategy for fusing data from different sources or subsets.

        Canonical Correlation Analysis (CCA) \cite{Hotelling} can be considered as the first approach of MvL with the aim to find pairs of projections for two views so that the correlations between these views are maximized.
        As CCA can only handle the linear correlation, Kernel CCA (KCCA) was proposed to take non-linear correlation relationship of data into account \cite{Akaho2006}.
        However, both CCA and KCCA are unsupervised methods and can not leverage the label information.
        In \cite{diethe2008multiview}, a supervised approach named Multi-view Fisher discriminant analysis (MvFDA) was proposed for binary classification problem.
        All of aforementioned methods are only applicable for two views problem.

        To extend to multiple view cases, a natural extension is to maximize the sum of the pairwise correlations.
        In a general case, it would be better to build a common shared feature space that captures latent information of the object from all observed views.
        For this propose, Multi-view CCA (MvCCA) is proposed in 2010 to build a common feature space of all views \cite{rupnik2010multi}. 
        %MCCA tried to find $v$ transformations by maximizing the correlation of every two views. 
        However, MvCCA did not consider the discrepancy information but only maximizing the correlation between every two views, so that it may be ineffective for classification across views. 
        %Generalized Multi-view Analysis (GMA) preserves the supervised structure of each view while keeping the projections of different views close to each other in the latent common space. GMA is considered as an extension of Fisher Discriminant analysis (FDA) for cross-view problem. It considers class label information so it could be good for multi-view classification. However, GMA considers only the the discriminant information in each individual view, not inter-view so it could decrease cross-view recognition.
        %Multi-view Uncorrelated Linear Discriminant Analysis (MULDA) \cite{sun2015multi-view} learnt uncorrelated discriminant features by using Uncorrelated Linear Discriminant Analysis (ULDA). Multi-view Modular Discriminant Analysis (MvMDA) \cite{cao2017generalized} was proposed to separate class centers across different views. 

        In \cite{kan2015multi}, Multi-view discriminant analysis (MvDA), an extension of linear discriminant analysis (LDA) for multi-view problem was proposed.
        MvDA tries to optimize jointly view correlation, intra-view and inter-view discriminability. 
        An extension of MvDA which considers view-consistency was also introduced and achieved significant performance improvement.

        In \cite{zhao2018multi}, the authors proposed multi-view manifold learning with locality alignment (MvML-LA) framework to realize manifold learning under multi-view scenario. 
        %Locality alignment in the latent space learning is considered to enhance its discriminative capability and developed two specific algorithms in supervised and unsupervised scenarios, respectively. 

        Most recently, \cite{you2019multi} proposed Multi-view Common Component Discriminant Analysis (MvCCDA) technique that both integrates supervised information and local geometric information into the common component extraction process.
        This helps to effectively handle view discrepancy, discriminability and non-linearity in a joint manner.


    \section{Summary}
        In this chapter, various related works were briefed.
        The general concepts of deep neural networks, especially 3D CNN because of its outstanding performance in action recognition for video data, are explained.
        Also, the underlying mathematical model behind LDA and two utilized multi-view analysis algorithms inspired by it (MvDA \& MvDA-vc) were clarified.
        In addition, an extension of LDA called pc-LDA that enhances it with better class discrepancy constraints were introduced.

    %!TEX root = ../../main.tex

\section{Related Works} \label{sec:related_works}

    \subsection{Human action and gesture recognition}
        Action recognition has been an attractive research topic since the last decade \cite{zhang2019comprehensive}.
        Early methods represented human actions by extracting 2D/3D key-points such as Harris-3D, SIFT-3D, HOG-3DHOF \cite{laptev2008learning}, ESURF \cite{willems2008efficient} then computed a descriptor from the detected key-points.
        Action representation  by a set of key-points could loose the temporal information. Therefore, Wang and Schmid in \cite{wang2013action} proposed a feature named improved dense trajectories (iDT) that densely sample and track optical flow points along trajectories.
        iDT has become state-of-the-art hand-crafted features and widely used for many video-based tasks.
        However, when working with large-scale datasets, iDT becomes intractable on due to its expensive computational cost and poor performance. 

        To work with more challenging datasets, effective action recognition approaches rely on powerful learning methods, particularly the deep learning techniques.
        Early works applied 2D CNN on frames of video sequence and then aggregated the information using pooling techniques \cite{karpathy2014large}.
        To exploit the temporal information, different architectures such as LSTM with the internal mechanisms called gates that can deal with short-term memory are proposed \cite{sun2017lattice}.
        Recently, instead of using 2D convolutional operators, different 3D CNN have been proposed \cite{ji20123d, tran2015learning, varol2017long}.
        Besides, to boost the recognition performance, different approaches tried to combine multiple streams \cite{wang2015towards, feichtenhofer2016convolutional, khong2018improving} or to combine both multiple features \cite{wang2015action, christoph2016spatiotemporal}. %trajectory-pooled deep-convolutional
        %descriptor - TDD;  ST-ResNet+iDT
        %Recently, many other architectures such as temporal Segment Networks - TSN \cite{wang2016temporal}, ST-VLMPF \cite{duta2017spatio}, P3D ResNet \cite{qiu2017learning}, I3D\cite{carreira2017quo}, 3D ResNeXt \cite{hara2018can}, R(2+1) D-TwoStream \cite{tran2018closer}, CO2FI+ASYN \cite{lin2018action}, and DML \cite{chen2017deep} have shown state-of-the-art performances in action recognition.
        %Previous section briefly gives a survey of different techniques for features extraction and action recognition from common views.

        These aforementioned approaches focus on single view action recognition, cross-view action recognition is more challenging and requires additional techniques to be taken into account. 
        Junejo et al. in \cite{junejo2008cross} proposed a descriptor, namely self-similarity matrix (SSM), which is an exhaustive table of distances between image features taken by pair from the image sequences.
        Liu et al. \cite{liu2011cross} employed cuboids extracted from each video and BoW model to build video descriptor for each single view. Then, a bipartite graph is built to model two view-dependent vocabularies.
        Li et al. \cite{li2012discriminative} described each video by concatenating spatio-temporal interest-point-based descriptor with shape flow descriptor. Then, to deal with cross-view, they construct ‘virtual views’, each is a linear transformation between action descriptors from one viewpoint and those from another.
        The method in \cite{zheng2012cross, zheng2013learning} employed the same video representation manner as in \cite{li2012discriminative}.
        However, a transferable dictionary between source and target view has been learnt to force features of the same action extracted from two views having the same sparse representation. 
        %In \cite{ulhaq2017space}, the authors proposed an advanced space-time filtering framework for recognizing human actions despite large viewpoint variations. Specifically, they used 3D tensor structure at each pixel, which characterizes the most common local motion in action sequences. Discrete tensor Fourier transform is then applied to achieve frequency domain representations. Then, they form view clusters from multiple-view action data and use space-time correlation filtering to achieve robust view representations. 

        Previous cross-view action recognition techniques usually connect source and target views with a set of linear transformations, that are unable to capture the non-linear manifolds on which real actions lie. In \cite{rahmani2017learning}, the authors find a shared high-level non-linear virtual path that connects multiple source and target views to the same canonical view. This virtual path is learnt by a deep neural network. In \cite{kong2017deeply}, a deep learning technique that stacks multiple layers of feature learners is designed to incorporate both private and shared view features. 
        In \cite{liu2018hierarchically}, the authors concatenated both private and shared view features and learnt transferable dictionary pair from a pair of views. In \cite{zhang2018action}, the authors proposed a framework to jointly learn a a view-invariant transfer dictionary and a view-invariant classifier using synthetic data during the pre-training phase to extract view-invariance between 3D and 2D videos.

        %As we have analyzed, multi-view  analysis has been actively studied. Several innovative ideas have been proposed. As a result, performance of multi-view classification is significantly improved. However, most of these methods are experimented on still images. The question of how still good those methods a special time-series data (i.e. video data) has been not addressed. In this paper, we contribute to improve the model of common feature space. In addition, we will evaluate the proposed method on video data, where both temporal and spatial features must be taken into account.

    \subsection{Multi-view analysis and learning techniques}
        As many objects in the real-world can be observed from different viewpoints, to exploit the consensual and complementary information between different views, Multi-view analysis (MvA) techniques are employed.
        MvA is a strategy for fusing data from different sources or subsets.

        Canonical Correlation Analysis (CCA) \cite{Hotelling} can be considered as the first approach of MvL with the aim to find pairs of projections for two views so that the correlations between these views are maximized.
        As CCA can only handle the linear correlation, Kernel CCA (KCCA) was proposed to take non-linear correlation relationship of data into account \cite{Akaho2006}.
        However, both CCA and KCCA are unsupervised methods and can not leverage the label information.
        In \cite{diethe2008multiview}, a supervised approach named Multi-view Fisher discriminant analysis (MvFDA) was proposed for binary classification problem.
        All of aforementioned methods are only applicable for two views problem.

        To extend to multiple view cases, a natural extension is to maximize the sum of the pairwise correlations.
        In a general case, it would be better to build a common shared feature space that captures latent information of the object from all observed views.
        For this propose, Multi-view CCA (MvCCA) is proposed in 2010 to build a common feature space of all views \cite{rupnik2010multi}. 
        %MCCA tried to find $v$ transformations by maximizing the correlation of every two views. 
        However, MvCCA did not consider the discrepancy information but only maximizing the correlation between every two views, so that it may be ineffective for classification across views. 
        %Generalized Multi-view Analysis (GMA) preserves the supervised structure of each view while keeping the projections of different views close to each other in the latent common space. GMA is considered as an extension of Fisher Discriminant analysis (FDA) for cross-view problem. It considers class label information so it could be good for multi-view classification. However, GMA considers only the the discriminant information in each individual view, not inter-view so it could decrease cross-view recognition.
        %Multi-view Uncorrelated Linear Discriminant Analysis (MULDA) \cite{sun2015multi-view} learnt uncorrelated discriminant features by using Uncorrelated Linear Discriminant Analysis (ULDA). Multi-view Modular Discriminant Analysis (MvMDA) \cite{cao2017generalized} was proposed to separate class centers across different views. 

        In \cite{kan2015multi}, Multi-view discriminant analysis (MvDA), an extension of linear discriminant analysis (LDA) for multi-view problem was proposed.
        MvDA tries to optimize jointly view correlation, intra-view and inter-view discriminability. 
        An extension of MvDA which considers view-consistency was also introduced and achieved significant performance improvement.

        In \cite{zhao2018multi}, the authors proposed multi-view manifold learning with locality alignment (MvML-LA) framework to realize manifold learning under multi-view scenario. 
        %Locality alignment in the latent space learning is considered to enhance its discriminative capability and developed two specific algorithms in supervised and unsupervised scenarios, respectively. 

        Most recently, \cite{you2019multi} proposed Multi-view Common Component Discriminant Analysis (MvCCDA) technique that both integrates supervised information and local geometric information into the common component extraction process.
        This helps to effectively handle view discrepancy, discriminability and non-linearity in a joint manner.


    \section{Summary}
        In this chapter, various related works were briefed.
        The general concepts of deep neural networks, especially 3D CNN because of its outstanding performance in action recognition for video data, are explained.
        Also, the underlying mathematical model behind LDA and two utilized multi-view analysis algorithms inspired by it (MvDA \& MvDA-vc) were clarified.
        In addition, an extension of LDA called pc-LDA that enhances it with better class discrepancy constraints were introduced.

    %!TEX root = ../../main.tex

\chapter{Technical Background and Related Works} \label{chap:background}
    \section{Introduction}
        This chapter provides the basic knowledge as well as related works regarding to the research topic of this thesis.
        Section \ref{sec:technical_background} introduces briefly the general architecture of deep neural networks for deep feature extraction; then describes in detail some dimensionality reduction algorithms and multi-view analysis algorithms.
        Section \ref{sec:related_works} summarizes approaches introduced in existing works to tackle problems in human action and gesture recognition and multi-view strategy.

    %!TEX root = ../../main.tex

\chapter{Technical Background and Related Works} \label{chap:background}
    \section{Introduction}
        This chapter provides the basic knowledge as well as related works regarding to the research topic of this thesis.
        Section \ref{sec:technical_background} introduces briefly the general architecture of deep neural networks for deep feature extraction; then describes in detail some dimensionality reduction algorithms and multi-view analysis algorithms.
        Section \ref{sec:related_works} summarizes approaches introduced in existing works to tackle problems in human action and gesture recognition and multi-view strategy.

    %!TEX root = ../../main.tex

\chapter{Technical Background and Related Works} \label{chap:background}
    \section{Introduction}
        This chapter provides the basic knowledge as well as related works regarding to the research topic of this thesis.
        Section \ref{sec:technical_background} introduces briefly the general architecture of deep neural networks for deep feature extraction; then describes in detail some dimensionality reduction algorithms and multi-view analysis algorithms.
        Section \ref{sec:related_works} summarizes approaches introduced in existing works to tackle problems in human action and gesture recognition and multi-view strategy.

    \input{sections/technical_background_and_related_works/technical_background/index}
    \input{sections/technical_background_and_related_works/related_works}

    \section{Summary}
        In this chapter, various related works were briefed.
        The general concepts of deep neural networks, especially 3D CNN because of its outstanding performance in action recognition for video data, are explained.
        Also, the underlying mathematical model behind LDA and two utilized multi-view analysis algorithms inspired by it (MvDA \& MvDA-vc) were clarified.
        In addition, an extension of LDA called pc-LDA that enhances it with better class discrepancy constraints were introduced.

    %!TEX root = ../../main.tex

\section{Related Works} \label{sec:related_works}

    \subsection{Human action and gesture recognition}
        Action recognition has been an attractive research topic since the last decade \cite{zhang2019comprehensive}.
        Early methods represented human actions by extracting 2D/3D key-points such as Harris-3D, SIFT-3D, HOG-3DHOF \cite{laptev2008learning}, ESURF \cite{willems2008efficient} then computed a descriptor from the detected key-points.
        Action representation  by a set of key-points could loose the temporal information. Therefore, Wang and Schmid in \cite{wang2013action} proposed a feature named improved dense trajectories (iDT) that densely sample and track optical flow points along trajectories.
        iDT has become state-of-the-art hand-crafted features and widely used for many video-based tasks.
        However, when working with large-scale datasets, iDT becomes intractable on due to its expensive computational cost and poor performance. 

        To work with more challenging datasets, effective action recognition approaches rely on powerful learning methods, particularly the deep learning techniques.
        Early works applied 2D CNN on frames of video sequence and then aggregated the information using pooling techniques \cite{karpathy2014large}.
        To exploit the temporal information, different architectures such as LSTM with the internal mechanisms called gates that can deal with short-term memory are proposed \cite{sun2017lattice}.
        Recently, instead of using 2D convolutional operators, different 3D CNN have been proposed \cite{ji20123d, tran2015learning, varol2017long}.
        Besides, to boost the recognition performance, different approaches tried to combine multiple streams \cite{wang2015towards, feichtenhofer2016convolutional, khong2018improving} or to combine both multiple features \cite{wang2015action, christoph2016spatiotemporal}. %trajectory-pooled deep-convolutional
        %descriptor - TDD;  ST-ResNet+iDT
        %Recently, many other architectures such as temporal Segment Networks - TSN \cite{wang2016temporal}, ST-VLMPF \cite{duta2017spatio}, P3D ResNet \cite{qiu2017learning}, I3D\cite{carreira2017quo}, 3D ResNeXt \cite{hara2018can}, R(2+1) D-TwoStream \cite{tran2018closer}, CO2FI+ASYN \cite{lin2018action}, and DML \cite{chen2017deep} have shown state-of-the-art performances in action recognition.
        %Previous section briefly gives a survey of different techniques for features extraction and action recognition from common views.

        These aforementioned approaches focus on single view action recognition, cross-view action recognition is more challenging and requires additional techniques to be taken into account. 
        Junejo et al. in \cite{junejo2008cross} proposed a descriptor, namely self-similarity matrix (SSM), which is an exhaustive table of distances between image features taken by pair from the image sequences.
        Liu et al. \cite{liu2011cross} employed cuboids extracted from each video and BoW model to build video descriptor for each single view. Then, a bipartite graph is built to model two view-dependent vocabularies.
        Li et al. \cite{li2012discriminative} described each video by concatenating spatio-temporal interest-point-based descriptor with shape flow descriptor. Then, to deal with cross-view, they construct ‘virtual views’, each is a linear transformation between action descriptors from one viewpoint and those from another.
        The method in \cite{zheng2012cross, zheng2013learning} employed the same video representation manner as in \cite{li2012discriminative}.
        However, a transferable dictionary between source and target view has been learnt to force features of the same action extracted from two views having the same sparse representation. 
        %In \cite{ulhaq2017space}, the authors proposed an advanced space-time filtering framework for recognizing human actions despite large viewpoint variations. Specifically, they used 3D tensor structure at each pixel, which characterizes the most common local motion in action sequences. Discrete tensor Fourier transform is then applied to achieve frequency domain representations. Then, they form view clusters from multiple-view action data and use space-time correlation filtering to achieve robust view representations. 

        Previous cross-view action recognition techniques usually connect source and target views with a set of linear transformations, that are unable to capture the non-linear manifolds on which real actions lie. In \cite{rahmani2017learning}, the authors find a shared high-level non-linear virtual path that connects multiple source and target views to the same canonical view. This virtual path is learnt by a deep neural network. In \cite{kong2017deeply}, a deep learning technique that stacks multiple layers of feature learners is designed to incorporate both private and shared view features. 
        In \cite{liu2018hierarchically}, the authors concatenated both private and shared view features and learnt transferable dictionary pair from a pair of views. In \cite{zhang2018action}, the authors proposed a framework to jointly learn a a view-invariant transfer dictionary and a view-invariant classifier using synthetic data during the pre-training phase to extract view-invariance between 3D and 2D videos.

        %As we have analyzed, multi-view  analysis has been actively studied. Several innovative ideas have been proposed. As a result, performance of multi-view classification is significantly improved. However, most of these methods are experimented on still images. The question of how still good those methods a special time-series data (i.e. video data) has been not addressed. In this paper, we contribute to improve the model of common feature space. In addition, we will evaluate the proposed method on video data, where both temporal and spatial features must be taken into account.

    \subsection{Multi-view analysis and learning techniques}
        As many objects in the real-world can be observed from different viewpoints, to exploit the consensual and complementary information between different views, Multi-view analysis (MvA) techniques are employed.
        MvA is a strategy for fusing data from different sources or subsets.

        Canonical Correlation Analysis (CCA) \cite{Hotelling} can be considered as the first approach of MvL with the aim to find pairs of projections for two views so that the correlations between these views are maximized.
        As CCA can only handle the linear correlation, Kernel CCA (KCCA) was proposed to take non-linear correlation relationship of data into account \cite{Akaho2006}.
        However, both CCA and KCCA are unsupervised methods and can not leverage the label information.
        In \cite{diethe2008multiview}, a supervised approach named Multi-view Fisher discriminant analysis (MvFDA) was proposed for binary classification problem.
        All of aforementioned methods are only applicable for two views problem.

        To extend to multiple view cases, a natural extension is to maximize the sum of the pairwise correlations.
        In a general case, it would be better to build a common shared feature space that captures latent information of the object from all observed views.
        For this propose, Multi-view CCA (MvCCA) is proposed in 2010 to build a common feature space of all views \cite{rupnik2010multi}. 
        %MCCA tried to find $v$ transformations by maximizing the correlation of every two views. 
        However, MvCCA did not consider the discrepancy information but only maximizing the correlation between every two views, so that it may be ineffective for classification across views. 
        %Generalized Multi-view Analysis (GMA) preserves the supervised structure of each view while keeping the projections of different views close to each other in the latent common space. GMA is considered as an extension of Fisher Discriminant analysis (FDA) for cross-view problem. It considers class label information so it could be good for multi-view classification. However, GMA considers only the the discriminant information in each individual view, not inter-view so it could decrease cross-view recognition.
        %Multi-view Uncorrelated Linear Discriminant Analysis (MULDA) \cite{sun2015multi-view} learnt uncorrelated discriminant features by using Uncorrelated Linear Discriminant Analysis (ULDA). Multi-view Modular Discriminant Analysis (MvMDA) \cite{cao2017generalized} was proposed to separate class centers across different views. 

        In \cite{kan2015multi}, Multi-view discriminant analysis (MvDA), an extension of linear discriminant analysis (LDA) for multi-view problem was proposed.
        MvDA tries to optimize jointly view correlation, intra-view and inter-view discriminability. 
        An extension of MvDA which considers view-consistency was also introduced and achieved significant performance improvement.

        In \cite{zhao2018multi}, the authors proposed multi-view manifold learning with locality alignment (MvML-LA) framework to realize manifold learning under multi-view scenario. 
        %Locality alignment in the latent space learning is considered to enhance its discriminative capability and developed two specific algorithms in supervised and unsupervised scenarios, respectively. 

        Most recently, \cite{you2019multi} proposed Multi-view Common Component Discriminant Analysis (MvCCDA) technique that both integrates supervised information and local geometric information into the common component extraction process.
        This helps to effectively handle view discrepancy, discriminability and non-linearity in a joint manner.


    \section{Summary}
        In this chapter, various related works were briefed.
        The general concepts of deep neural networks, especially 3D CNN because of its outstanding performance in action recognition for video data, are explained.
        Also, the underlying mathematical model behind LDA and two utilized multi-view analysis algorithms inspired by it (MvDA \& MvDA-vc) were clarified.
        In addition, an extension of LDA called pc-LDA that enhances it with better class discrepancy constraints were introduced.

    %!TEX root = ../../main.tex

\section{Related Works} \label{sec:related_works}

    \subsection{Human action and gesture recognition}
        Action recognition has been an attractive research topic since the last decade \cite{zhang2019comprehensive}.
        Early methods represented human actions by extracting 2D/3D key-points such as Harris-3D, SIFT-3D, HOG-3DHOF \cite{laptev2008learning}, ESURF \cite{willems2008efficient} then computed a descriptor from the detected key-points.
        Action representation  by a set of key-points could loose the temporal information. Therefore, Wang and Schmid in \cite{wang2013action} proposed a feature named improved dense trajectories (iDT) that densely sample and track optical flow points along trajectories.
        iDT has become state-of-the-art hand-crafted features and widely used for many video-based tasks.
        However, when working with large-scale datasets, iDT becomes intractable on due to its expensive computational cost and poor performance. 

        To work with more challenging datasets, effective action recognition approaches rely on powerful learning methods, particularly the deep learning techniques.
        Early works applied 2D CNN on frames of video sequence and then aggregated the information using pooling techniques \cite{karpathy2014large}.
        To exploit the temporal information, different architectures such as LSTM with the internal mechanisms called gates that can deal with short-term memory are proposed \cite{sun2017lattice}.
        Recently, instead of using 2D convolutional operators, different 3D CNN have been proposed \cite{ji20123d, tran2015learning, varol2017long}.
        Besides, to boost the recognition performance, different approaches tried to combine multiple streams \cite{wang2015towards, feichtenhofer2016convolutional, khong2018improving} or to combine both multiple features \cite{wang2015action, christoph2016spatiotemporal}. %trajectory-pooled deep-convolutional
        %descriptor - TDD;  ST-ResNet+iDT
        %Recently, many other architectures such as temporal Segment Networks - TSN \cite{wang2016temporal}, ST-VLMPF \cite{duta2017spatio}, P3D ResNet \cite{qiu2017learning}, I3D\cite{carreira2017quo}, 3D ResNeXt \cite{hara2018can}, R(2+1) D-TwoStream \cite{tran2018closer}, CO2FI+ASYN \cite{lin2018action}, and DML \cite{chen2017deep} have shown state-of-the-art performances in action recognition.
        %Previous section briefly gives a survey of different techniques for features extraction and action recognition from common views.

        These aforementioned approaches focus on single view action recognition, cross-view action recognition is more challenging and requires additional techniques to be taken into account. 
        Junejo et al. in \cite{junejo2008cross} proposed a descriptor, namely self-similarity matrix (SSM), which is an exhaustive table of distances between image features taken by pair from the image sequences.
        Liu et al. \cite{liu2011cross} employed cuboids extracted from each video and BoW model to build video descriptor for each single view. Then, a bipartite graph is built to model two view-dependent vocabularies.
        Li et al. \cite{li2012discriminative} described each video by concatenating spatio-temporal interest-point-based descriptor with shape flow descriptor. Then, to deal with cross-view, they construct ‘virtual views’, each is a linear transformation between action descriptors from one viewpoint and those from another.
        The method in \cite{zheng2012cross, zheng2013learning} employed the same video representation manner as in \cite{li2012discriminative}.
        However, a transferable dictionary between source and target view has been learnt to force features of the same action extracted from two views having the same sparse representation. 
        %In \cite{ulhaq2017space}, the authors proposed an advanced space-time filtering framework for recognizing human actions despite large viewpoint variations. Specifically, they used 3D tensor structure at each pixel, which characterizes the most common local motion in action sequences. Discrete tensor Fourier transform is then applied to achieve frequency domain representations. Then, they form view clusters from multiple-view action data and use space-time correlation filtering to achieve robust view representations. 

        Previous cross-view action recognition techniques usually connect source and target views with a set of linear transformations, that are unable to capture the non-linear manifolds on which real actions lie. In \cite{rahmani2017learning}, the authors find a shared high-level non-linear virtual path that connects multiple source and target views to the same canonical view. This virtual path is learnt by a deep neural network. In \cite{kong2017deeply}, a deep learning technique that stacks multiple layers of feature learners is designed to incorporate both private and shared view features. 
        In \cite{liu2018hierarchically}, the authors concatenated both private and shared view features and learnt transferable dictionary pair from a pair of views. In \cite{zhang2018action}, the authors proposed a framework to jointly learn a a view-invariant transfer dictionary and a view-invariant classifier using synthetic data during the pre-training phase to extract view-invariance between 3D and 2D videos.

        %As we have analyzed, multi-view  analysis has been actively studied. Several innovative ideas have been proposed. As a result, performance of multi-view classification is significantly improved. However, most of these methods are experimented on still images. The question of how still good those methods a special time-series data (i.e. video data) has been not addressed. In this paper, we contribute to improve the model of common feature space. In addition, we will evaluate the proposed method on video data, where both temporal and spatial features must be taken into account.

    \subsection{Multi-view analysis and learning techniques}
        As many objects in the real-world can be observed from different viewpoints, to exploit the consensual and complementary information between different views, Multi-view analysis (MvA) techniques are employed.
        MvA is a strategy for fusing data from different sources or subsets.

        Canonical Correlation Analysis (CCA) \cite{Hotelling} can be considered as the first approach of MvL with the aim to find pairs of projections for two views so that the correlations between these views are maximized.
        As CCA can only handle the linear correlation, Kernel CCA (KCCA) was proposed to take non-linear correlation relationship of data into account \cite{Akaho2006}.
        However, both CCA and KCCA are unsupervised methods and can not leverage the label information.
        In \cite{diethe2008multiview}, a supervised approach named Multi-view Fisher discriminant analysis (MvFDA) was proposed for binary classification problem.
        All of aforementioned methods are only applicable for two views problem.

        To extend to multiple view cases, a natural extension is to maximize the sum of the pairwise correlations.
        In a general case, it would be better to build a common shared feature space that captures latent information of the object from all observed views.
        For this propose, Multi-view CCA (MvCCA) is proposed in 2010 to build a common feature space of all views \cite{rupnik2010multi}. 
        %MCCA tried to find $v$ transformations by maximizing the correlation of every two views. 
        However, MvCCA did not consider the discrepancy information but only maximizing the correlation between every two views, so that it may be ineffective for classification across views. 
        %Generalized Multi-view Analysis (GMA) preserves the supervised structure of each view while keeping the projections of different views close to each other in the latent common space. GMA is considered as an extension of Fisher Discriminant analysis (FDA) for cross-view problem. It considers class label information so it could be good for multi-view classification. However, GMA considers only the the discriminant information in each individual view, not inter-view so it could decrease cross-view recognition.
        %Multi-view Uncorrelated Linear Discriminant Analysis (MULDA) \cite{sun2015multi-view} learnt uncorrelated discriminant features by using Uncorrelated Linear Discriminant Analysis (ULDA). Multi-view Modular Discriminant Analysis (MvMDA) \cite{cao2017generalized} was proposed to separate class centers across different views. 

        In \cite{kan2015multi}, Multi-view discriminant analysis (MvDA), an extension of linear discriminant analysis (LDA) for multi-view problem was proposed.
        MvDA tries to optimize jointly view correlation, intra-view and inter-view discriminability. 
        An extension of MvDA which considers view-consistency was also introduced and achieved significant performance improvement.

        In \cite{zhao2018multi}, the authors proposed multi-view manifold learning with locality alignment (MvML-LA) framework to realize manifold learning under multi-view scenario. 
        %Locality alignment in the latent space learning is considered to enhance its discriminative capability and developed two specific algorithms in supervised and unsupervised scenarios, respectively. 

        Most recently, \cite{you2019multi} proposed Multi-view Common Component Discriminant Analysis (MvCCDA) technique that both integrates supervised information and local geometric information into the common component extraction process.
        This helps to effectively handle view discrepancy, discriminability and non-linearity in a joint manner.


    \section{Summary}
        In this chapter, various related works were briefed.
        The general concepts of deep neural networks, especially 3D CNN because of its outstanding performance in action recognition for video data, are explained.
        Also, the underlying mathematical model behind LDA and two utilized multi-view analysis algorithms inspired by it (MvDA \& MvDA-vc) were clarified.
        In addition, an extension of LDA called pc-LDA that enhances it with better class discrepancy constraints were introduced.

    %!TEX root = ../main.tex

\cleardoublepage
\ihead[]{Appendix}
\ohead[]{\pagemark}
\begin{appendix}
\chapter{Appendix} \label{chap:appendix}

\section{Derivation} \label{app:derivation}

This chapter supplies the expansion formulas of equations regarding multi-view analysis algorithms mentioned in this thesis, including MvDA, MvDA-vc and the proposed pc-MvDA.

For easy follow-up, let us briefly reduplicate the definitions and notations that will be used.
$X = \left[X_1, X_2, ..., X_v\right] = \{x_{ijk}|i=(1,..,c);j = (1,..,v);k=(1,..,n_{ij})\}$ and $Y = \left[Y_1, Y_2, ..., Y_v\right] = \{y_{ijk} = w_j^T x_{ijk}|i=(1,..,c); j=(1,..,v); k=(1,...,n_{ij})\}$ stand for the $v$-view dataset of $c$ classes and $n$ samples before and after projection respectively.
The mean of dataset is designated with $\mu = \frac{1}{n}\sum_{i=1}^{c}\sum_{j=1}^{v}\sum_{k=1}^{n_{ij}}y_{ijk}$, mean of class $\mu_{i} = \frac{1}{n_i}\sum_{j=1}^{v}\sum_{k=1}^{n_{ij}}y_{ijk}$ and mean of class in one view $\mu_{ij} = \frac{1}{n_{ij}}\sum_{k=1}^{n_{ij}}y_{ijk}$.
$W = \left[\omega_1, \omega_2, ..., \omega_v\right]$ are $v$ transformations learnt for each view by solving an optimization problem that minimizes within-class scatter matrix $\boldsymbol{S}^y_W$ and maximizes between-class scatter matrix $\boldsymbol{S}^y_B$.
Here the ubiquitous subscripts $i, a, b$ denote class, $j, r$ denote view and $k$ denotes index; while the less-frequently used superscripts $x$ and $y$ allude the original or transformed common dimension respectively of the corresponding term.

Supposing data samples from all views are aligned identically, in essence, the $k^{th}$ sample of $X_j$ is the common component of $k^{th}$ sample of $X_r$ $\forall j, r \in (1, 2, ..., v)$.
In addition to the aformentioned notations, we define the class vector $e_i \in \mathbb{R}^{\frac{n}{v}\times 1}$ of class $i$, which has $k^{th}$ element ${e_i}_{(k)} = 1$ if $class(x_k) = i$ and ${e_i}_{(k)} = 0$ otherwise.
It follows that $e = \sum_{i=1}^{c}e_i$ is a vector of ones as each sample strictly belongs to one class.

By using class vector $e_i$ as mask over $X_j$, mean of class $i$ in original dimension of view $j$ can be expressed by:
\begin{equation}
    \mu^{(x)}_{ij} = \frac{1}{n_{ij}}\sum_{k=1}^{n_{ij}}x_{ijk} = \frac{1}{n_{ij}}X_j e_i
    \label{eq:mean_from_class_vector}
\end{equation}

\subsection{Derivation of \texorpdfstring{$\boldsymbol{S}^y_W$}{intra-class} and \texorpdfstring{$\boldsymbol{S}^y_B$}{inter-class} scatter matrices in MvDA} \label{subsec:derivation_mvda}
    The expansions of $\boldsymbol{S}^y_W$ and $\boldsymbol{S}^y_B$ used in this thesis slightly differ from those supplemented in the original publication \cite{kan2015multi} in order to derive the pre-transformed version $\boldsymbol{S}^x_W$ and $\boldsymbol{S}^x_B$ and thus requires extra reformulation from the steps where we can subtitute \eqref{eq:mean_from_class_vector} in.

    The within-class scatter matrix $\boldsymbol{S}^y_W$ of MvDA in Equation \eqref{eq:MvDA_Sw} is expanded as follows:
    \begin{equation}
        \begin{split}
            \boldsymbol{S}^y_W &= \sum_{i=1}^{c}\sum_{j=1}^{v}\sum_{k=1}^{n_{ij}}(y_{ijk}-\mu_i)(y_{ijk}-\mu_i)^T \\
            &= \sum_{i=1}^{c}\sum_{j=1}^{v}\sum_{k=1}^{n_{ij}}\left(y_{ijk}y_{ijk}^T - y_{ijk}\mu_i^T - \mu_iy_{ijk}^T + \mu_i\mu_i^T\right) \\
            &= \sum_{i=1}^{c}\left(\sum_{j=1}^{v}\sum_{k=1}^{n_{ij}}y_{ijk}y_{ijk}^T - n_i\mu_i\mu_i^T\right) \\
            &= \sum_{i=1}^{c}\left(\sum_{j=1}^{v}\sum_{k=1}^{n_{ij}}y_{ijk}y_{ijk}^T - \frac{1}{n_i}\left(\sum_{j=1}^{v}n_{ij}\mu_{ij}\right){\left(\sum_{j=1}^{v}n_{ij}\mu_{ij}\right)}^T\right) \\
            &= \sum_{i=1}^{c}\left(\sum_{j=1}^{v}\sum_{k=1}^{n_{ij}}y_{ijk}y_{ijk}^T - \frac{1}{n_i}\sum_{j=1}^{v}\sum_{r=1}^{v}n_{ij}n_{ir}\mu_{ij}\mu_{ir}^T\right) \\
            &= \sum_{j=1}^{v}\omega_j^T\left(\sum_{i=1}^{c}\sum_{k=1}^{n_{ij}}x_{ijk}x_{ijk}^T\right)\omega_j - \sum_{j=1}^{v}\sum_{r=1}^{v}\omega_j^T\left(\sum_{i=1}^{c}\frac{n_{ij}n_{ir}}{n_i}\mu^{(x)}_{ij}{\mu^{(x)}_{ir}}^T\right)\omega_r \\
            &= \sum_{j=1}^{v}\omega_j^T X_j I X_j^T\omega_j - \sum_{j=1}^{v}\sum_{r=1}^{v}\omega_j^T\left(\sum_{i=1}^{c}\frac{n_{ij}n_{ir}}{n_i}\left(\frac{1}{n_{ij}}X_j e_i\right)\left(\frac{1}{n_{ir}}X_r e_i\right)^T\right)\omega_r \\
            &= \sum_{j=1}^{v}\omega_j^T X_j I X_j^T\omega_j - \sum_{j=1}^{v}\sum_{r=1}^{v}\omega_j^T\left(\sum_{i=1}^{c}\frac{1}{n_i}X_j e_i e_i^T X_r^T\right)\omega_r \\
            &= \sum_{j=1}^{v}\omega_j^T X_j I X_j^T\omega_j - \sum_{j=1}^{v}\sum_{r=1}^{v}\omega_j^T X_j\left(\sum_{i=1}^{c}\frac{1}{n_i}e_i e_i^T\right)X_r^T\omega_r \\
            &= \sum_{j=1}^{v}\omega_j^T X_j I X_j^T\omega_j - \sum_{j=1}^{v}\sum_{r=1}^{v}\omega_j^T X_j E X_r^T\omega_r \\
            &= W^T X \boldsymbol{I} X^T W - W^T X \boldsymbol{E} X^T W \\
            &= W^T X \left(\boldsymbol{I} - \boldsymbol{E}\right) X^T W \\
            \Rightarrow \boldsymbol{S}^x_W &= X \left(\boldsymbol{I} - \boldsymbol{E}\right) X^T
        \end{split}
        \label{eq:mvda_Sw_derivation}
    \end{equation}
    where $I \in \mathbb{R}^{\frac{n}{v}\times \frac{n}{v}}$ and $\boldsymbol{I} \in \mathbb{R}^{n\times n}$ are identity matrices; $E = \sum_{i=1}^{c}\frac{1}{n_i}e_i e_i^T \in \mathbb{R}^{\frac{n}{v}\times \frac{n}{v}}$ is a square matrix whose elements satisfy:
    \begin{equation}
        \boldsymbol{E}_{(k,l)} = \left\{\begin{array}{lr}
            \frac{1}{n_i}, & \text{if } class(x_k) = class(x_l) = i\\
            0, & \text{otherwise}
            \end{array}\right\}
    \end{equation}
    and $\boldsymbol{E} = \left[E\right]_{v\times v} \in \mathbb{R}^{n\times n}$ is $v \times v$ grid stack of $E$:
    \begin{equation}
        \boldsymbol{E} = \left[\begin{matrix}E&E&\cdots&E\\E&E&\cdots&E\\\vdots&\vdots&\ddots&\vdots\\E&E&\cdots&E\\\end{matrix}\right]
    \end{equation}

    Using the distributivity identity of summation, it is easy to prove that:
    \begin{equation}
        \sum_{a=1}^{c}\sum_{b=1}^{c}e_a e_b^T = \left(\sum_{i=1}^{c}e_i\right)\left(\sum_{i=1}^{c}e_i^T\right) = ee^T
    \end{equation}
    where $e$ is a vector of ones, hence, its self product results in a square matrix of ones.
    Then, the between-class scatter matrix $\boldsymbol{S}^y_B$ of MvDA in Equation \eqref{eq:MvDA_Sb} can be expanded as follows:
    \begin{equation}
        \begin{split}
            \boldsymbol{S}^y_B &= \sum_{i=1}^{c}n_i\left(\mu_i - \mu\right)\left(\mu_i - \mu\right)^T \\
            &= \sum_{i=1}^{c}n_i\left(\mu_i\mu_i^T - \mu_i\mu^T - \mu\mu_i^T + \mu\mu^T\right) \\
            &= \sum_{i=1}^{c}n_i\mu_i\mu_i^T - n\mu\mu^T \\
            &= \sum_{i=1}^{c}n_i\left(\frac{1}{n_i}\sum_{j=1}^{v}n_{ij}\mu_{ij}\right)\left(\frac{1}{n_i}\sum_{j=1}^{v}n_{ij}\mu_{ij}\right)^T - n\left(\frac{1}{n}\sum_{i=1}^{c}\sum_{j=1}^{v}n_{ij}\mu_{ij}\right)\left(\frac{1}{n}\sum_{i=1}^{c}\sum_{j=1}^{v}n_{ij}\mu_{ij}\right)^T \\
            &= \sum_{i=1}^{c}\frac{1}{n_i}\left(\sum_{j=1}^{v}\sum_{r=1}^{v}n_{ij}n_{ir}\mu_{ij}\mu_{ir}^T\right) - \frac{1}{n}\left(\sum_{j=1}^{v}\sum_{i=1}^{c}n_{ij}\mu_{ij}\right)\left(\sum_{j=1}^{v}\sum_{i=1}^{c}n_{ij}\mu_{ij}\right)^T \\
            &= \sum_{j=1}^{v}\sum_{r=1}^{v}\left(\sum_{i=1}^{c}\frac{n_{ij}n_{ir}}{n_i}\mu_{ij}\mu_{ir}^T\right) - \frac{1}{n}\sum_{j=1}^{v}\sum_{r=1}^{v}\left(\sum_{i=1}^{c}n_{ij}\mu_{ij}\right)\left(\sum_{i=1}^{c}n_{ij}\mu_{ij}\right)^T \\
            &= \sum_{j=1}^{v}\sum_{r=1}^{v}\left(\sum_{i=1}^{c}\frac{n_{ij}n_{ir}}{n_i}\mu_{ij}\mu_{ir}^T\right) - \frac{1}{n}\sum_{j=1}^{v}\sum_{r=1}^{v}\left(\sum_{a=1}^{c}\sum_{b=1}^{c}n_{aj}n_{br}\mu_{aj}\mu_{br}^T\right) \\
            &= \sum_{j=1}^{v}\sum_{r=1}^{v}\omega_j^T\left(\sum_{i=1}^{c}\frac{n_{ij}n_{ir}}{n_i}\mu^{(x)}_{ij}{\mu^{(x)}_{ir}}^T\right)\omega_r - \sum_{j=1}^{v}\sum_{r=1}^{v}\omega_j^T\left(\sum_{a=1}^{c}\sum_{b=1}^{c}\frac{n_{aj}n_{br}}{n}\mu^{(x)}_{aj}{\mu^{(x)}_{br}}^T\right)\omega_r \\
            &= \sum_{j=1}^{v}\sum_{r=1}^{v}\omega_j^T\left(\sum_{i=1}^{c}\frac{n_{ij}n_{ir}}{n_i}\left(\frac{1}{n_{ij}}X_j e_i\right)\left(\frac{1}{n_{ir}}X_r e_i\right)^T\right)\omega_r \\
            &\ \ \ \ \ \ \ \ \ \ \ \ \ \ \ \ \ \ \ \ \ \ \ \ \ \ \ \ \ \  - \sum_{j=1}^{v}\sum_{r=1}^{v}\omega_j^T\left(\sum_{a=1}^{c}\sum_{b=1}^{c}\frac{n_{aj}n_{br}}{n}\left(\frac{1}{n_{aj}}X_j e_a\right)\left(\frac{1}{n_{br}}X_r e_b\right)\right)\omega_r \\
            &= \sum_{j=1}^{v}\sum_{r=1}^{v}\omega_j^T X_j\left(\sum_{i=1}^{c}\frac{1}{n_i}e_i e_i^T\right)X_r^T\omega_r - \sum_{j=1}^{v}\sum_{r=1}^{v}\omega_j^T X_j\left(\sum_{a=1}^{c}\sum_{b=1}^{c}\frac{1}{n}e_a e_b^T\right)X_r^T\omega_r \\
            &= \sum_{j=1}^{v}\sum_{r=1}^{v} \omega_j^T X_j E X_r^T \omega_r - \sum_{j=1}^{v}\sum_{r=1}^{v} \omega_j^T X_j \frac{1}{n}\mathbbm{1} X_r^T \omega_r \\
            &= W^T X \boldsymbol{E} X^T W - W^T X \frac{1}{n}\boldsymbol{\mathbbm{1}} X^T W \\
            &= W^T X \left(\boldsymbol{E} - \frac{1}{n}\boldsymbol{\mathbbm{1}}\right) X^T W \\
            \Rightarrow \boldsymbol{S}^x_B &= X \left(\boldsymbol{E} - \frac{1}{n}\boldsymbol{\mathbbm{1}}\right) X^T
        \end{split}
        \label{eq:mvda_Sb_derivation}
    \end{equation}
    where $E = \sum_{i=1}^{c}\frac{1}{n_i}e_i e_i^T \in \mathbb{R}^{\frac{n}{v}\times \frac{n}{v}}$ and $\boldsymbol{E} = \left[E\right]_{v\times v} \in \mathbb{R}^{n\times n}$ are defined above; $\mathbbm{1} = \sum_{a=1}^{c}\sum_{b=1}^{c}e_a e_b^T = \left[1\right]_{\frac{n}{v} \times \frac{n}{v}} \in \mathbb{R}^{\frac{n}{v}\times \frac{n}{v}}$ and $\boldsymbol{\mathbbm{1}} = \left[\mathbbm{1}\right]_{v\times v} = \left[1\right]_{n \times n} \in \mathbb{R}^{n\times n}$ are matrices of ones.

\subsection{Derivation of \texorpdfstring{$\boldsymbol{O}_{view-consistency}$}{view-consistency term} in MvDA-vc} \label{subsec:derivation_mvdavc}

    The view-consistency objective of MvDA-vc in Equation \eqref{eq:MvDA-vc_vc} is expanded as follows:
    \begin{equation}
        \begin{split}
            \boldsymbol{O}_{view-consistency} &= \sum_{j=1}^{v}\sum_{r=1}^{v}\left|\left|\boldsymbol{\beta}_j - \boldsymbol{\beta}_r\right|\right|_2^2 \\
            &= \sum_{j=1}^{v}\sum_{r=1}^{v}\left(\boldsymbol{\beta}_j^T\boldsymbol{\beta}_j - \boldsymbol{\beta}_j^T\boldsymbol{\beta}_r - \boldsymbol{\beta}_r^T\boldsymbol{\beta}_j + \boldsymbol{\beta}_r^T\boldsymbol{\beta}_r\right) \\
            &= \sum_{j=1}^{v}\sum_{r=1}^{v}\left(2\boldsymbol{\beta}_j^T\boldsymbol{\beta}_j - 2\boldsymbol{\beta}_j^T\boldsymbol{\beta}_r\right) \\
            &= \sum_{j=1}^{v}2v\omega_j^T\boldsymbol{P}_j I \boldsymbol{P}_j^T\omega_j - \sum_{j=1}^{v}\sum_{r=1}^{v}2\omega_j^T\boldsymbol{P}_j I \boldsymbol{P}_r^T\omega_r \\
            &= W^T \boldsymbol{P}^T \left(2v\boldsymbol{I}\right) \boldsymbol{P} W - W^T \boldsymbol{P}^T \left(2\boldsymbol{\widehat{I}}\right) \boldsymbol{P} W \\
            &= W^T \boldsymbol{P}^T \left(2v\boldsymbol{I} - 2\boldsymbol{\widehat{I}}\right) \boldsymbol{P} W
        \end{split}
        \label{eq:mvdavc_vc_derivation}
    \end{equation}
    where $\boldsymbol{\beta}_j = \boldsymbol{P}_j\omega_j$ and $\boldsymbol{P}_j = \left(X_j^T X_j\right)^{-1}X_j^T$ as defined in Section \ref{subsubsec:mvdavc}; $I \in \mathbb{R}^{\frac{n}{v}\times \frac{n}{v}}$ and $\boldsymbol{I} \in \mathbb{R}^{n\times n}$ are defined in \ref{subsec:derivation_mvda}, $\boldsymbol{\widehat{I}} = \left[I\right]_{v\times v} \in \mathbb{R}^{n\times n}$ is $v\times v$ grid stack of $I$:
    \begin{equation}
        \boldsymbol{\widehat{I}} = \left[\begin{matrix}I&I&\cdots&I\\I&I&\cdots&I\\\vdots&\vdots&\ddots&\vdots\\I&I&\cdots&I\\\end{matrix}\right]
    \end{equation}

\subsection{Derivation of \texorpdfstring{${\boldsymbol{S}^x_W}_{ab}$}{paired intra-class} and \texorpdfstring{${\boldsymbol{S}^x_B}_{ab}$}{paired inter-class} scatter matrices in pc-MvDA} \label{subsec:derivation_pcmvda}

    Firstly we reformulate the local intra-class ${\boldsymbol{S}^x_W}_{i}$ from Equation \eqref{eq:pcmvda_Sw_i}.
    It can be easily derived by removing sum over classes $\sum_{i=1}^{c}$ from $\boldsymbol{S}_W^y$ in Equation \eqref{eq:mvda_Sw_derivation}:
    \begin{equation}
        \begin{split}
            {\boldsymbol{S}_W^y}_i &= \sum_{j=1}^{v}\sum_{k=1}^{n_{ij}}\left(y_{ijk}-\mu_i\right)\left(y_{ijk}-\mu_i\right)^T \\
            &= \sum_{j=1}^{v}\omega_j^T\left(\sum_{k=1}^{n_ij}x_{ijk}x_{ijk}^T\right)\omega_j - \sum_{j=1}^{v}\sum_{r=1}^{v}\omega_j^T X_j\left(\frac{1}{n_i} e_i e_i^T\right)X_r^T\omega_r \\
            &= \sum_{j=1}^{v}\omega_j^T X_j I_i X_j^T\omega_j - \sum_{j=1}^{v}\sum_{r=1}^{v}\omega_j^T X_j E X_r^T\omega_r \\
            &= W^T X \boldsymbol{I}_i X^T W - W^T X \boldsymbol{E} X^T W \\
            &= W^T X \left(\boldsymbol{I}_i - \boldsymbol{E}\right) X^T W \\
            \Rightarrow {\boldsymbol{S}_W^x}_i &= X \left(\boldsymbol{I}_i - \boldsymbol{E}\right) X^T
        \end{split}
        \label{eq:pcmvda_Sw_i_derivation}
    \end{equation}
    where $E \in \mathbb{R}^{\frac{n}{v}\times \frac{n}{v}}$ and $\boldsymbol{E} \in \mathbb{R}^{n\times n}$ are defined in Section \ref{subsec:derivation_mvda}; $I_{i} = I e_i \in \mathbb{R}^{\frac{n}{v}\times \frac{n}{v}}$ is $e_i$ masked identity matrix whose elements satisfy:
    \begin{equation}
        {\boldsymbol{I}_{i}}_{(k,k)} = \left\{\begin{array}{lr}
            1, & \text{if } class(x_k) = i \\
            0, & \text{otherwise}
            \end{array}\right\}
    \end{equation}
    and $\boldsymbol{I}_{i} \in \mathbb{R}^{n\times n}$ is $v$ diagonal stack of $I_i$:
    \begin{equation}
        \boldsymbol{I}_{i} = \left[\begin{matrix}I_i&0&\cdots&0\\0&I_i&\cdots&0\\\vdots&\vdots&\ddots&\vdots\\0&0&\cdots&I_i\\\end{matrix}\right]
    \end{equation}

    Subtituing \eqref{eq:pcmvda_Sw_i_derivation} and \eqref{eq:mvda_Sw_derivation} in Equation \eqref{eq:pcmvda_Sw_ab} we get the paired intra-class ${\boldsymbol{S}_W^y}_{ab}$ of pc-MvDA:
    \begin{equation}
        \begin{split}
            {\boldsymbol{S}_W^y}_{ab} &= \beta\frac{n_a{\boldsymbol{S}_W^y}_a + n_b{\boldsymbol{S}_W^y}_b}{n_a + n_b} + (1 - \beta)\boldsymbol{S}_W^y \\
            &= \beta\frac{n_a W^T X \left(\boldsymbol{I}_a - \boldsymbol{E}\right) X^T W + n_b W^T X \left(\boldsymbol{I}_b - \boldsymbol{E}\right) X^T W}{n_a + n_b} \\
            &\ \ \ \ \ \ \ \ \ \ \ \ \ \ \ \ \ \ \ \ \ \ \ \ \ \ \ \ \ \ + (1 - \beta)W^T X \left(\boldsymbol{I} - \boldsymbol{E}\right) X^T W \\
            &= W^T X \left(\beta\frac{n_a\left(\boldsymbol{I}_a - \boldsymbol{E}\right) + n_b\left(\boldsymbol{I}_b - \boldsymbol{E}\right)}{n_a + n_b} + (1 - \beta)\left(\boldsymbol{I} - \boldsymbol{E}\right)\right) X^T W \\
            &= W^T X \left(\beta\frac{n_a\boldsymbol{I}_a + n_b\boldsymbol{I}_b}{n_a + n_b} + (1 - \beta)\boldsymbol{I} - \boldsymbol{E}\right) X^T W \\
           \Rightarrow {\boldsymbol{S}_W^x}_{ab} &= X \left(\beta\frac{n_a\boldsymbol{I}_a + n_b\boldsymbol{I}_b}{n_a + n_b} + (1 - \beta)\boldsymbol{I} - \boldsymbol{E}\right) X^T
        \end{split}
        \label{eq:pcmvda_Sw_ab_derivation}
    \end{equation}
    where $0 \leq \beta \leq 1$ is a hyperparameter.

    And finally, the paired between-class scatter matrix ${\boldsymbol{S}_B^y}_{ab}$ in Equation \eqref{eq:pcmvda_Sb_ab} is expanded as follows:
    \begin{equation}
        \begin{split}
            {\boldsymbol{S}_B^y}_{ab} &= {\left(\mu_a-\mu_b\right)\left(\mu_a-\mu_b\right)^T} \\
            &= \left[\sum_{j=1}^{v}\left(\frac{n_{aj}}{n_a}\mu_{aj} - \frac{n_{bj}}{n_b}\mu_{bj}\right)\right] \left[\sum_{j=1}^{v}\left(\frac{n_{aj}}{n_a}\mu_{aj} - \frac{n_{bj}}{n_b}\mu_{bj}\right)\right]^T \\
            &= \left[\sum_{j=1}^{v}\omega_j^T\left(\frac{n_{aj}}{n_a}\mu^{(x)}_{aj} - \frac{n_{bj}}{n_b}\mu^{(x)}_{bj}\right)\right] \left[\sum_{j=1}^{v}\omega_j^T\left(\frac{n_{aj}}{n_a}\mu^{(x)}_{aj} - \frac{n_{bj}}{n_b}\mu^{(x)}_{bj}\right)\right]^T \\
            &= \left[\sum_{j=1}^{v}\omega_j^T\left(\frac{n_{aj}}{n_a}\left(\frac{1}{n_{aj}}X_j e_a\right) - \frac{n_{bj}}{n_b}\left(\frac{1}{n_{bj}}X_j e_b\right)\right)\right] \\
            &\ \ \ \ \ \ \ \ \ \ \ \ \ \ \ \ \ \ \ \ \ \ \ \ \ \ \ \ \ \ \left[\sum_{j=1}^{v}\omega_j^T\left(\frac{n_{aj}}{n_a}\left(\frac{1}{n_{aj}}X_j e_a\right) - \frac{n_{bj}}{n_b}\left(\frac{1}{n_{bj}}X_j e_b\right)\right)\right]^T \\
            &= \left[\sum_{j=1}^{v}\omega_j^T X_j\left(\frac{1}{n_a}e_a - \frac{1}{n_b}e_b\right)\right] \left[\sum_{j=1}^{v}\omega_j^T X_j\left(\frac{1}{n_a}e_a - \frac{1}{n_b}e_b\right)\right]^T \\
            &= \sum_{j=1}^{v}\sum_{r=1}^{v}\omega_j^T X_j \left(\frac{1}{n_a}e_a - \frac{1}{n_b}e_b\right) \left(\frac{1}{n_a}e_a^T - \frac{1}{n_b}e_b^T\right) X_r^T\omega_r \\
            &= \sum_{j=1}^{v}\sum_{r=1}^{v}\omega_j^T X_j \left(\frac{1}{n_a^2}e_a e_a^T + \frac{1}{n_b^2}e_b e_b^T\right) X_r^T\omega_r \\
            &\ \ \ \ \ \ \ \ \ \ \ \ \ \ \ \ \ \ \ \ \ \ \ \ \ \ \ \ \ \ - \sum_{j=1}^{v}\sum_{r=1}^{v}\omega_j^T X_j \frac{1}{n_a n_b}\left(e_a e_b^T + e_b e_a^T\right) X_r^T\omega_r \\
            &= \sum_{j=1}^{v}\sum_{r=1}^{v}\omega_j^T X_j E_{ab} X_r^T\omega_r - \sum_{j=1}^{v}\sum_{r=1}^{v}\omega_j^T X_j \tilde{E}_{ab} X_r^T\omega_r \\
            &= W^T X \boldsymbol{E}_{ab} X^T W - W^T X \boldsymbol{\tilde{E}}_{ab} X^T W \\
            &= W^T X \left(\boldsymbol{E}_{ab} - \boldsymbol{\tilde{E}}_{ab}\right) X^T W \\
        \Rightarrow {\boldsymbol{S}_B^x}_{ab} &= X \left(\boldsymbol{E}_{ab} - \boldsymbol{\tilde{E}}_{ab}\right) X^T
        \end{split}
        \label{eq:pcmvda_Sb_i_derivation}
    \end{equation}
    where $E_{ab} = \frac{1}{n_a^2}e_a e_a^T + \frac{1}{n_b^2}e_b e_b^T \in \mathbb{R}^{\frac{n}{v}\times \frac{n}{v}}$ and $\tilde{E}_{ab} = \frac{1}{n_a n_b}\left(e_a e_b^T + e_b e_a^T\right) \in \mathbb{R}^{\frac{n}{v}\times \frac{n}{v}}$ are square matrices whose elements satisfy:
    \begin{align}
        {\boldsymbol{E}_{ab}}_{(k,l)} &= \left\{\begin{array}{lr}
            \frac{1}{{n_i}^2}, & \text{if } class(x_k) = class(x_l) = i \text{ and } i \in \{a, b\} \\
            0, & \text{otherwise}
            \end{array}\right\} \\
        {\boldsymbol{\tilde{E}}_{ab}}_{(k,l)} &= \left\{\begin{array}{lr}
            \frac{1}{n_a n_b}, & \text{if } i = class(x_k) \neq class(x_l) = j \text{ and } i,j \in \{a, b\} \\
            0, & \text{otherwise}
            \end{array}\right\}
    \end{align}
    and $\boldsymbol{E}_{ab} = \left[E_{ab}\right]_{v \times v} \in \mathbb{R}^{n\times n}$ and $\boldsymbol{\tilde{E}}_{ab} = \left[\tilde{E}_{ab}\right]_{v \times v} \in \mathbb{R}^{n\times n}$ are $v \times v$ grid stacks of $E_{ab}$ and $\tilde{E}_{ab}$ respectively:
    \begin{equation}
        \boldsymbol{E}_{ab} = \left[\begin{matrix}E_{ab}&E_{ab}&\cdots&E_{ab}\\E_{ab}&E_{ab}&\cdots&E_{ab}\\\vdots&\vdots&\ddots&\vdots\\E_{ab}&E_{ab}&\cdots&E_{ab}\\\end{matrix}\right]; \quad
        \boldsymbol{\tilde{E}}_{ab} = \left[\begin{matrix}\tilde{E}_{ab}&\tilde{E}_{ab}&\cdots&\tilde{E}_{ab}\\\tilde{E}_{ab}&\tilde{E}_{ab}&\cdots&\tilde{E}_{ab}\\\vdots&\vdots&\ddots&\vdots\\\tilde{E}_{ab}&\tilde{E}_{ab}&\cdots&\tilde{E}_{ab}\\\end{matrix}\right]
    \end{equation}

\normalsize
\end{appendix}


    \cleardoublepage
    \footnotesize
    \ihead[]{Bibliography}
    \bibliography{main}
    \normalsize

\end{document}
